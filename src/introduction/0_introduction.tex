% !TEX root = ../../thesis.tex
%
\chapter{Introduction}
\label{sec:intro}

\minitoc

\section{General context and motivations}

Almost a decade ago, \emph{quality assurance} (QA) was not a
common practice in most software companies, and researchers had
to prove, for instance, what the benefits of testing could be.
Generally speaking, quality assurance is a way of preventing
faults in manufactured products, which is defined in ISO 9000 as
\enquote{part of quality management focused on providing confidence that
quality requirements will be fulfilled} \cite{iso20059000}.
\emph{Software testing} provides information about the quality of
software. For instance, \enquote{testing can prove confidence of
a software system by executing a program with the intent of
finding errors} \cite{Myers:1979:AST:539883}, also known as
"bugs".

Nowadays, quality assurance and software testing are well-known
in the Industry, and everyone understands the need for them.
Yet, testing is often performed by hand, which is complicated and
far from perfect. Software testing is often seen as a sequence of
three major steps: \enquote{(i) the design of test cases that
are good at revealing faults, according to a certain level of
requirement \cite{kaner2003good}, (ii) the execution of these
test cases, and (iii) the determination of whether the
produced output is correct \cite{weyuker1982testing}}
\cite{lakhotia2009search}.

Sadly, the test case execution is often the only fully automated
aspect of this activity in the Industry. \emph{Continuous
Integration} (CI) \cite{booch1991object} is now associated with
the automation of the execution of test cases and quick feedback,
often received by email. There are countless tools and services
\footnote{For example, Travis CI: \url{https://travis-ci.org/}.}
to automate this process, but too few tools have emerged to
tackle the problem of the automatic test case generation.
Fortunately, in Academia, researchers have studied such a problem
for decades \cite{4221614}.

A relatively recent field to automate and improve testing is
\emph{Model-based Testing} (MbT). While the original idea has
been around for decades \cite{moore56}, there has been a growing
interest over the last years.  MbT is application of (formal)
Model-based design for designing and optionally also executing
artifacts to perform software testing
\cite{Jorgensen:1995:STC:526521}. The use of a model allows to
formally describe the expected behaviors of a software, from
which it is possible to automatically generate test cases,
and then to execute them.  Nonetheless, writing such models is
tedious and error-prone, which is a drawback and can also explain
the slow adoption of MbT in the Industry.

\emph{Model inference} is a research field that aims at
automatically deriving models, expressing functional behaviors of
existing software. These models, even if incomplete, help
understand how a software behaves. That is why model inference is
interesting to solve the limitation of MbT mentioned previously.
Most of the model inference techniques that are used for testing
purpose are not designed to infer large models or to test large
software systems though. Anti-Model-based Testing \cite{antimbt} is
somehow related to this idea, although Anti-Model-based Testing
is more about using testing to reverse-engineer a model, and then
check such a model to detect whether the software behaves
correctly.

Michelin, one of the three largest tire manufacturers in the
world, has been our industrial partner for the last three years.
The company designs most of its factories, production machines,
and software itself. In a factory, there are several workshops
for the different parts of the manufacturing process. From our
point of view, a workshop is seen as a set of production machines
controlled by a software. That is what we call a \emph{production
system}. Such systems deal with many physical devices, databases,
and also human interactions. Most of these systems run for years,
up to 20 years according to our partner. Maintaining and updating
such legacy software is complicated as documentation is often
outdated, and the developers who wrote these software are not
available anymore.  In this thesis, we propose a solution to the
problem of testing such systems called \textit{Autofunk}, a
framework (and a tool) combining different techniques to: (i)
infer formal models from large (legacy) production systems, and
(ii) perform Model-based Testing (using the inferred models) in
order to detect potential regressions.

The next section discusses in detail the problems this thesis
addresses. Section \ref{sec:intro:contrib} presents the
contributions of this thesis. Finally, Section
\ref{sec:intro:overview} gives the overview of this thesis.

%%%%%%%%%%%%%%%%%%%%%%%%%%%%%%%%%%%%%%%%%%%%%%%%%%%%%%%%%%%%%%%%

\section{Problems and objectives}
\label{sec:intro:problems}

Michelin is a tire manufacturer since 1889. With a strong but old
culture of secret, the company uses to design most of its
factories, production machines, and software itself. The
engineering department is dedicated to the construction of these
items, and gathers embedded software engineers, control
engineers, mechanical engineers, and also software engineers. The
computing sub-department deals with software that have been built
for decades, and we can count about 50 different applications,
and even more when we consider the different versions with
customization made for the different factories. For the record,
Michelin owns factories in more than 70 countries.

Different programming languages have been used for building such
software, as well as different development frameworks, and
various paradigms. This context leads to a large and disparate
set of legacy applications that have to be maintained, but also
updated with new features. As most of these software are built
for controlling and supervising a set of production machines in a
factory, they are considered critical in the sense that they can
break the production flow. The main issue faced by the software
engineers is the introduction of regressions. A \emph{regression}
is a software fault that makes a feature stop functioning as
intended after a certain event, for instance, the deployment of a
new version of a software. Regressions have to be revealed as
soon as possible in order to avoid production downtimes.

The goal of this thesis is to propose technical solutions to
Michelin's engineers to prevent such regressions with testing,
\emph{i.e.} performing \emph{regression testing} on Michelin's
production systems based on behavioral models of such systems.
Such models have to be up to date, hence we cannot rely on
existing documentation.  We determine, by means of Model-based
Testing, whether a change in one part of a software affects other
parts of it. To avoid disturbing the system, we use a
\emph{passive testing} approach, which only observes the
production system under test and does not interact with it.  The
main assumption in this work is that we choose to consider a
running production system in a production environment (that is, a
factory) as a fully operational system, \emph{i.e.} a system that
behaves correctly.  Such a hypothesis has been expressed by our
partner Michelin. The inferred models can be employed for
different purposes, \emph{e.g.}, documentation and testing, but
it is manifest that they should not be used for conformance
testing in general. Last but not least, we propose scalable
techniques because production systems exchange thousands
\emph{production events} a day.

In this thesis, we provide solutions for making Michelin's
production systems more reliable by means of testing, to ease
software engineers' work, thanks to two main lines:

\begin{itemize}
    \item The inference of partial yet exact models of production
        systems in a fast and efficient manner, based on the data
        exchanged in a (production) environment. A method to
        extract knowledge from data available in a production
        environment has to be defined before inferring models;

    \item The design of a conformance testing technique based on
        the inferred models, targeting production systems. It is
        worth mentioning that it is possible to perform
        conformance testing here thanks to the assumption
        introduced previously. Finally, due to the nature of
        these systems, the testing method should scale.
\end{itemize}

%%%%%%%%%%%%%%%%%%%%%%%%%%%%%%%%%%%%%%%%%%%%%%%%%%%%%%%%%%%%%%%%

\section{Contributions of the thesis}
\label{sec:intro:contrib}

The contributions of this thesis are:

\begin{enumerate}
    \item A preliminary study introducing our model inference
        framework, called \emph{Autofunk (v1)}, on web
        applications. This framework is framed upon a rule-based
        expert system capturing knowledge of human business
        experts. \textit{Autofunk} is also combined with an
        automatic testing technique to interact with web
        applications in order to improve the completeness of the
        inferred models.  Several models can be built at
        different levels of abstraction, allowing to create, for
        instance, human-readable documentation. This preliminary
        work targeting web applications validates our ideas and
        concepts to infer models from data recorded in a
        (production) environment;

    \item \textit{Autofunk (v2)}, the enhanced version of our
        framework, now combining several techniques originating
        from different fields such as expert systems, machine
        learning and model inference, for inferring formal models
        of legacy software systems. Our main goal is to infer
        models of Michelin's production systems with a
        \emph{black-box} approach. We choose to leverage the data
        exchanged among the devices and software in order to
        target most of the existing Michelin applications without
        having to consider the programming languages or
        frameworks used to develop any of these applications.
        \textit{Autofunk}'s modular architecture allows multiple
        extensions so that we can take different data sources as
        input, and perform various tasks based on the inferred
        models, \emph{e.g.}, testing.

    \item An evaluation of the model inference of Michelin's
        production systems with \textit{Autofunk}, proving that
        it is able to build exact models in an efficient manner,
        based on large sets of traces. For now, we define a set
        of traces as information collected from a production
        system, and describing its behaviors. A more formal
        definition is given in the sequel of this thesis.
        Exactness of our inferred models is defined by the
        trace-equivalence and trace-inclusion relations proposed
        in \cite{petrenko06}. Results show that our model
        inference technique scales well;

    \item A reduction technique for symbolic transition systems
        that is both fast and efficient, keeping the exactness of
        the models, and targeting large models. This technique
        has successfully been applied to large Michelin's
        production systems. This context-specific reduction
        technique is a key element to enable testing of
        Michelin's production systems because it improves the
        usability of the inferred models, which also favors the
        adoption of \emph{Autofunk} at Michelin;

    \item An offline passive testing technique leveraging the
        inferred models for testing production systems, along
        with a case study. This work is an extension of
        \textit{Autofunk} introducing two implementation
        relations: (i) the first relation refers to the trace
        preorder relation \cite{DNH84}, and (ii) the second
        relation is used to overcome a limitation caused by the
        partialness of the inferred models. This testing
        extension detects differences between two production
        systems, \emph{i.e.} potential regressions.
\end{enumerate}

\paragraph{Publications.} Most of these contributions have been
published in the following international conference proceedings:

\begin{itemize}
    \item \emph{Actes de la 13eme {\'e}dition d'AFADL, atelier
        francophone sur les Approches Formelles dans l'Assistance au
        D{\'e}veloppement de Logiciels} (AFADL'14)
        \cite{durand2014inference};

    \item Proceedings of the Fifth Symposium on Information and
        Communication Technology (SoICT'14)
        \cite{DBLP:conf/soict/DurandS14};

    \item Proceedings of Formal Methods 2015 (FM'15)
        \cite{DBLP:conf/fm/DurandS15};

    \item Proceedings of the 9th International Conference on
        Distributed Event-Based Systems (DEBS'15)
        \cite{DBLP:conf/debs/SalvaD15};

    \item Proceedings of the 13th International Conference on Formal
        Methods and Models for Co-Design (MEMOCODE'15)
        \cite{7340480}.
\end{itemize}

%%%%%%%%%%%%%%%%%%%%%%%%%%%%%%%%%%%%%%%%%%%%%%%%%%%%%%%%%%%%%%%%

\section{Overview of the thesis}
\label{sec:intro:overview}

The work presented in this thesis deals with two main research
realms: (i) (software) model inference, and (ii) software
testing. Hence, Chapter \ref{sec:related} is split into two main
sections: Section \ref{sec:related:testing} introduces important
notions of software testing, and Section
\ref{sec:related:modelinf} reviews the literature on software
model inference techniques. Chapters \ref{sec:modelinf:webapps}
and \ref{sec:modelinf:prodsystems} are dedicated to the model
inference of software systems (web applications first, and
production systems then). Chapter \ref{sec:testing} is dedicated
to the testing of production systems. Chapter
\ref{sec:conclusion} ends this thesis with conclusions and
perspectives for future work. We give more details about each
chapter below.

\textbf{Chapter \ref{sec:related}} surveys the literature in
software testing first, and then in model inference applied to
software systems. The chapter starts by introducing what software
testing means, mentioning some important notions as well as the
different types of testing. It then presents what Model-based
Testing (MbT) is, along with a few definitions and common terms
employed in MbT. The software testing part ends with a review on
some passive testing techniques, and why they are interesting in
our case. The second part of this chapter presents what model
inference is, from active inference, \emph{i.e.} methods
interacting with a software system to extract knowledge about it,
to passive inference, \emph{i.e.} techniques that infer models
from a fixed set of knowledge, such as a set of execution traces,
source code, or even documentation.

\textbf{Chapter \ref{sec:modelinf:webapps}} presents our work on
model inference based on the remarks made in Chapter
\ref{sec:related}. This chapter introduces a preliminary work on
model inference of web applications that gave birth to
\textit{Autofunk}, our modular framework for inferring models
(and later, performing testing). This chapter gives an overview
of \textit{Autofunk}'s very first architecture, called
\emph{Autofunk v1}. It then presents how \emph{Autofunk} relies
on an automatic testing technique to extract more knowledge about
the targeted web applications. This is important to infer more
complete models.  A note on its implementation is
given, following by an experimentation. This work has been
published in \emph{Actes de la 13eme {\'e}dition d’AFADL, atelier
francophone sur les Approches Formelles dans l’Assistance au
D{\'e}veloppement de Logiciels} (AFADL'14)
\cite{durand2014inference}, and in the Proceedings of the Fifth
Symposium on Information and Communication Technology (SoICT'14)
\cite{DBLP:conf/soict/DurandS14}.

\textbf{Chapter \ref{sec:modelinf:prodsystems}} introduces our
framework \textit{Autofunk} revisited to target production
systems. This chapter gives the context that led to our choices
regarding the design of \textit{Autofunk v2}. Our \emph{reduction
technique} that heavily reduces models is then presented, along
with the results of experiments on Michelin's production systems.
A whole section is dedicated to the implementation of
\textit{Autofunk} for Michelin. Last significant part of this
chapter is the use of a machine learning technique to maximize
one step of our model inference technique, which led to the
creation of \emph{Autofunk v3}. This work has been published in
the Proceedings of Formal Methods 2015 (FM'15)
\cite{DBLP:conf/fm/DurandS15}, and in the Proceedings of the 9th
International Conference on Distributed Event-Based Systems
(DEBS'15) \cite{DBLP:conf/debs/SalvaD15}.

\textbf{Chapter \ref{sec:testing}} tackles the problem of
passively testing production systems, without disturbing them,
and without having any specification. It presents our work on
offline passive testing, by extending \textit{Autofunk v3}'s
model inference framework. After having presented the overall
idea, our passive algorithm is given and explained. The results
of experiments on Michelin's production systems are also given.
This work has been published in the Proceedings of the 13th
International Conference on Formal Methods and Models for
Co-Design (MEMOCODE'15) \cite{7340480}.

\textbf{Chapter \ref{sec:conclusion}} closes the main body of the
thesis with concluding comments and proposals for future work.

% The End.
\cleardoublepage
\blankpage
