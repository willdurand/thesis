% !TEX root = ../../thesis.tex
%
\chapter{Introduction}
\label{sec:intro}

\minitoc

I am studying Computer Science for more than 8 years now. I
remember my first lectures on quality assurance (QA) and software
testing, how pointless it seemed to all of us, students who don't
write bugs \footnote{The last part of this sentence is adapted
from this tweet:
\url{https://twitter.com/hipsterhacker/status/396352411754717184}.}.
Indeed, almost a decade ago, quality assurance was not a common
practice, and researchers had to prove what the benefits of, for
instance, testing could be.

Nowadays, quality assurance and software testing are well-known
in the Industry, and everyone understands the need for them.

\TODO{...}


\section{Contributions}

The contributions of this thesis are:

\begin{enumerate}
    \item \textit{Autofunk}, a framework, combining several
        techniques originating from different fields such as
        expert systems, machine learning and model inference, for
        inferring formal models of software systems. Autofunk's
        modular architecture allows multiple extensions;

    \item An empirical study that evaluates Autofunk on web
        applications for inferring models. Here, Autofunk is
        combined with an automatic testing technique to interact
        with web applications in order to improve the
        completeness of the inferred models. Several models can
        be built at different levels of abstraction, allowing to
        create, for instance, human-readable documentation;

    \item An empirical study that evaluates Autofunk on
        Michelin's production systems, proving that it is able to
        build exact models in an efficient manner, based on large
        sets of traces (for now, we can define a set of traces as
        information collected from a software system, and
        describing its behaviors);

    \item A reduction technique for symbolic transition systems
        that is both fast and efficient, keeping the exactness of
        the models, and targetting large models;

    \item An offline passive testing technique leveraging the
        inferred models for testing production systems, along
        with a case study. This work is an extension of Autofunk;

    \item An online passive testing technique similar to the
        offline passive testing, but enabling on the fly testing
        of production systems.
\end{enumerate}


\section{Overview}

\textbf{Chapter \ref{sec:related}} surveys the literature in
software testing first, and then in model inference applied to
software systems. The chapter starts by introducing what software
testing means, mentioning some important notions as well as the
different types of testing. It then presents what Model-based
Testing (MbT) is, along with a few definitions and common terms
employed in MbT. The software testing part ends with a review on
some passive testing techniques, and why they are interesting in
our case. The second part of this chapter presents what model
inference is, from active learning to passive learning
techniques.

\textbf{Chapter \ref{sec:modelinf:webapps}} presents our work on
model inference applied to web applications, a preliminary work
that gave birth to \textit{Autofunk}, our modular framework for
inferring models (and later, performing testing). This chapter
gives an overview of Autofunk's very first architecture. It then
presents how Autofunk relies on an automatic testing technique to
improve the completeness of the inferred models. A note on its
implementation is then given, following by an experimentation.


\textbf{Chapter \ref{sec:modelinf:prodsystems}} introduces our
framework Autofunk revisited to target production systems. This
chapter gives the context that led to our choices regarding the
edesign of Autofunk. Our reduction technique that heavily reduces
models is then presented, along with the results of a case study
on Michelin's production systems. A whole section is dedicated to
the implementation of Autofunk for Michelin. Last significant
part of this chapter is the use of a machine learning technique
to maximize one part of our model inference technique.

\textbf{Chapter \ref{sec:testing}} tackles the problem of testing
production systems, without disturbing them, and without having
any specification. It presents our work on both offline and
online passive testing, by extending Autofunk's model inference
ramework. After having presented the overall idea, two algorithms
are given and explained. The results of a case study on
Michelin's production systems is given. Such results are related
to the offline passive testing technique.

\textbf{Chapter \ref{sec:conclusion}} closes the main body of the
thesis with concluding comments and proposals for future work.

% The End.
\cleardoublepage
\blankpage
