\section{Introduction}
\label{sec:testing:intro}

Manual testing is, by far, the most popular technique for
testing, but this technique is known to be error-prone as well.
As already formulated in this thesis, production systems are
usually composed of thousands of states (\emph{i.e.} sets of
conditions that exist at a given instant in time) and production
events, which makes testing time consuming.  In this context, we
propose a \emph{passive testing framework} for production systems
that is composed of two parts: a model inference engine, already
presented in Chapter \ref{sec:modelinf:prodsystems}, and a
passive test engine that is the purpose of this chapter. Both
parts have to be fast and scalable to be used in practice.

The main idea of our proposal is that, given a running production
system, we extract knowledge and models by passively monitoring
it. Such models describe the functional behaviors of the system,
and may serve for different purposes, \emph{e.g.}, testing
another production system. The latter can be a new system roughly
comparable to the first one in terms of features, but it can also
be an updated version of the first one. Indeed, upgrades might
inadvertently introduce faults, and it could lead to severe
damages. Here, testing the updated system means detecting
potential regressions before deploying changes in production.

A \textit{passive tester} (also known as observer) aims at
checking whether a system under test \emph{conforms to} an
inferred model. It can be performed in either online or offline
mode, as defined below:

\begin{itemize}
    \item \textbf{Online testing:} sometimes called on-the-fly
        testing, it is a technique in which test case generation
        and test execution are combined into a single algorithm.
        The tester gives a verdict every time a trace incomes;

    \item \textbf{Offline testing:} it means that a set of traces
        has been collected while the system is running. Then, the
        tester gives verdicts.
\end{itemize}

We collect the traces of the system under test by reusing
\textit{Autofunk v3}'s Models generator. In offline mode, we
build a set of traces with the same level of abstraction as those
considered for inferring models. In online mode, we apply the
same process for each new incoming trace. Then, we use these
traces to check if the system under test conforms to the inferred
models. \emph{Conformance} is defined with two implementation
relations, which express precisely what the system under test
should do. The first relation corresponds to the \emph{trace
preorder} \cite{DNH84}, which is a well-known relation based upon
trace inclusion, and heavily used with passive testing.
Nevertheless, our inferred models are partials, \emph{i.e.} they
do not necessarily capture all the possible behaviors that should
happen. That is why we propose a second implementation relation,
less restrictive on the traces that should be observed from the
system under test.

Both offline and online modes are not completely unalike, they
also serve different purposes. In previous works, we used to work
with fixed sets of traces. We noticed that, by taking large trace
sets, we could build more complete models. Performing offline
passive testing allows to use such large trace sets, hence the
intuition that we should be able to validate more behaviors. Our
online passive testing approach records traces on a system under
test on-the-fly, and then check whether those traces satisfy
specifications, still generated from a system under analysis. It
enables what we call \emph{just-in-time fault detection}. Faults
can be revealed in near real-time on a running system so that
users can be notified as soon as possible.

In Section \ref{sec:testing:normal}, we introduce an extra step
of the model inference method described in Chapter
\ref{sec:modelinf:prodsystems} that is required to enable
testing. In Section \ref{sec:testing:passive}, we present both
our offline and online passive testing techniques, built on-top of
this model inference framework, respectively in Section
\ref{sec:testing:offline} and Section \ref{sec:testing:online}.
We present key results on offline passive testing in Section
\ref{sec:testing:offline:impl-exp}. Finally, we conclude on this
chapter in Section \ref{sec:testing:conclusion}.

\textbf{Publication.} This work has been partially published in
the Proceedings of the 13th International Conference on Formal
Methods and Models for Co-Design (MEMOCODE'15) \cite{7340480}.

%%%%%%%%%%%%%%%%%%%%%%%%%%%%%%%%%%%%%%%%%%%%%%%%%%%%%%%%%%%%%%%%%

\section{Extending inferred models with normalization}
\label{sec:testing:normal}

In order to perform testing, we reuse the reduced model
$R(\EuScript{S}) = \{R(\EuScript{S}_1),\dots,R(\EuScript{S}_n)\}$
inferred with \textit{Autofunk} that we \textit{normalize} to get
rid of some runtime-dependent information.  Indeed, both models
$\EuScript{S}$ and $R(\EuScript{S})$ include parameters that are
dependent to the products being manufactured.  That is a
consequence of generating models that describe behaviors of a
continuous stream of products that are strictly identified,
\emph{i.e.} for each action in a given sequence, we have the
assignment $(pid := val)$ (for the record, $pid$ stands for
\emph{product identifier}).  Here, we normalize these models
before using them for testing. The resulting models are denoted
by $\EuScript{S}^{N} = \{ \EuScript{S}_1^{N}, \dots,
\EuScript{S}_n^{N} \}$ and $R(\EuScript{S}^{N}) = \{
R(\EuScript{S}_1^{N}), \dots, R(\EuScript{S}_n^{N}) \}$.

We remove the assignments relative to product identifiers and
time stamps. Furthermore, we label all the final locations with
"Pass". We flag these locations as \emph{verdict locations}, and
gather them in the set $Pass \subseteq L_{\EuScript{S_i}^{N}}$.
Both $\EuScript{S}^{N}$ and $R(\EuScript{S}^{N})$ represent more
generic models, \emph{i.e.}  they express \textit{some possible
behaviors that should happen}. These behaviors are represented by
the traces $Traces_{Pass} (\EuScript{S}^{N})=\displaystyle
\bigcup_{1 \leq i \leq n} Traces_{Pass}
(\EuScript{S}_i^{N})=Traces_{Pass} (R(\EuScript{S}^{N}))$. We
refer to these traces as \textit{pass traces}, and we call the
other traces \textit{possibly fail traces}.

%%%%%%%%%%%%%%%%%%%%%%%%%%%%%%%%%%%%%%%%%%%%%%%%%%%%%%%%%%%%%%%%%

\section{Passive testing with \textit{Autofunk}}
\label{sec:testing:passive}

We consider both models $\EuScript{S}^{N}$ and
$R(\EuScript{S}^{N})$ of a system under analysis $\mathit{Sua}$,
generated by our inference-based model generation framework, as
\emph{reference models}. In this section, we present the second
part of our framework, dedicated to the passive testing of a
system under test $\mathit{Sut}$.

Figure \ref{fig:passive-autofunk} depicts the architecture of our
passive testing technique. In offline mode, a set of production
events has been collected beforehand from $\mathit{Sut}$ in the
same way as for $\mathit{Sua}$. These events are grouped into
traces to form the trace set $Traces({Sut})$, and then filtered
to obtain a set of complete traces denoted by $CTraces({Sut})$
(cf.
\crossref{sec:modelinf:prodsystems}{sec:modelinf:prodsystems:better-segmentation}).
In online mode (not represented on the figure), we do not have
traces but rather filtered valued events of $\mathit{Sut}$ coming
on-the-fly. Finally, we perform passive testing to \emph{check}
if $\mathit{Sut}$ conforms to $\EuScript{S}^{N}$.

\begin{figure}[h]
    \begin{center}
        \includegraphics[width=1.0\linewidth]{figures/passive_autofunk.png}
    \end{center}

    \caption{The architecture of \textit{Autofunk v3} with the
    passive testing extension. While the previous
    \textit{Autofunk}'s architecture has been kept, there are
    two new modules: "STS Normalization" and "check",
    representing the passive conformance testing part.}
    \label{fig:passive-autofunk}
\end{figure}

Our industrial partner wishes to check whether every complete
execution trace of $\mathit{Sut}$ matches a behavior captured by
$\EuScript{S}^{N}$. In this case, the test verdict must reflect a
successful result. On the contrary, if an execution of
$\mathit{Sut}$ is not captured by $\EuScript{S}^{N}$, one cannot
conclude that $\mathit{Sut}$ is faulty because $\EuScript{S}^{N}$
is a partial model, and it does not necessarily includes all the
correct behaviors. Below, we formalize theses verdict notions
with two implementation relations. Such relations between models
can only be written by assuming the following \emph{test
assumption}: the black-box system $\mathit{Sut}$ can be described
by a model, here with a Labeled Transition System as defined in
\crossref{sec:related:testing}{sec:definitions:lts}. We also
denote this model by $\mathit{Sut}$.

The first implementation relation, denoted by $\leq_{ct}$, refers
to the trace preorder relation
\cite{DNH84,vaandrager1991relationship} (cf. Example
\ref{example:trace_preorder} on page
\pageref{example:trace_preorder}).
It aims at checking whether all the complete execution traces of
$\mathit{Sut}$ are pass traces of
$\EuScript{S}^{N}=\{\EuScript{S}_1^{N},\dots,\EuScript{S}_n^{N}\}$.
The first implementation relation can be written with the
following definition:

\begin{definition}[The $\leq_{ct}$ implementation relation]
\label{rel:impl1}

Let $\EuScript{S}^{N}$ be an inferred model of $\mathit{Sua}$, and
$\mathit{Sut}$ be the system under test. When $\mathit{Sut}$
produces complete traces also captured by $\EuScript{S}^{N}$, we
write: ${Sut} \leq_{ct} \EuScript{S}^{N} =_{def} CTraces({Sut})
\subseteq  Traces_{Pass}(\EuScript{S}^{N})$ \cite{Tre96}.
\end{definition}

The reduced model $R(\EuScript{S}^{N})$ sounds more convenient
for passively testing $\mathit{Sut}$ because it is strongly
reduced in terms of size compared to $\EuScript{S}^{N}$.  Hence,
the test relation can also be written as below because both models
$\EuScript{S}^{N}$ and $R(\EuScript{S}^{N})$ are trace
equivalent \cite{petrenko06}:

\begin{proposition}
\label{rel:impl12}
${Sut} \leq_{ct} \EuScript{S}^{N} \text{ if and only if } CTraces({Sut})
\subseteq  Traces_{Pass}(R(\EuScript{S}^{N}))$.
\end{proposition}

As stated previously, the inferred model $\EuScript{S}^{N}$ of
$\mathit{Sua}$ is partial, and it might not capture all the
behaviors that should happen on $\mathit{Sut}$. Consequently,
our partner wants a \emph{weaker implementation relation} that is
less restrictive on the traces that should be observed from
$\mathit{Sut}$.  In other words, this second relation aims to
check that, for every complete trace $t=a_1(\alpha_1) \dots
a_m(\alpha_m)$ of $\mathit{Sut}$, we also have a set of traces of
$Traces_{Pass}(\EuScript{S}^{N})$ having the same sequence of
symbols such that every variable assignment $\alpha_j(x)_{(1 \leq
j \leq m)}$ of $t$ is found in one of the traces of
$Traces_{Pass}(\EuScript{S}^{N})$ with the same symbol $a_j$.

\begin{example}
    Figure \ref{fig:sts-ch4} recalls the example considered in
    Chapter \ref{sec:modelinf:prodsystems}. The trace $t =
    17011(nsys:=1, nsec:=8, point:=1, pid:=1)\text{ }
    17021(nsys:=1, nsec:=8, point:=4, tpoint:=9, pid:=1)$ is not
    a \emph{pass trace} of $\EuScript{S}^{N}$ because this trace
    cannot be extracted from one of the paths of the STS depicted
    in Figure \ref{fig:sts-ch4}, on account of the variables
    $point$ and $tpoint$, which do not take the expected values.
    Nonetheless, both variables are assipned with $(point := 4)$
    and $(tpoint := 9)$ in the second path. This is interesting
    as it indicates that such values \emph{may} be correct since
    they are actually used in a similar action in a similar path.
    Our second implementation relation aims at expressing that
    such a trace $t$ captures a correct behavior as well.

    \begin{figure}[ht]
        \begin{center}
            \includegraphics[width=1.0\linewidth]{figures/STS1.png}
        \end{center}

        \caption{The first Symbolic Transition System inferred in
        Chapter \ref{sec:modelinf:prodsystems}.}
        \label{fig:sts-ch4}
    \end{figure}
\end{example}

The second implementation relation, denoted by $\leq_{mct}$, is
defined as follows:

\begin{definition}[The $\leq_{mct}$ implementation relation]
     Let $\EuScript{S}^{N}$ be an inferred model of
     $\mathit{Sua}$, and $\mathit{Sut}$ be a system under test.
     We write: ${Sut} \leq_{mct} \EuScript{S}^{N}$ if and only if
     $\forall ~t= a_1(\alpha_1) \dots a_m(\alpha_m) \in
     CTraces({Sut})$, $\forall ~\alpha_j(x)_{(1 \leq j \leq m)},
     \exists ~\EuScript{S}_i^{N} \in \EuScript{S}^{N}$ and $t'
     \in Traces_{Pass}(\EuScript{S}_i^{N})$ such that
     $t'=a_1(\alpha_1') \dots a_m(\alpha_m')$ and
     $\alpha_j'(x)=\alpha_j(x)$.

     \label{impl21}
\end{definition}

According to the above definition, the successive symbols and
variable assignments of a trace $t \in CTraces({Sut})$ must be
found into several traces of $Traces_{Pass}(\EuScript{S}_i^{N})$,
which have the same sequence of symbols $a_1 \dots a_m$ as the
trace $t$. The reduced model $R(\EuScript{S}_i^{N})$ was
previously constructed to capture all these traces in
$Traces_{Pass}(\EuScript{S}_i^{N})$, having the same sequence of
symbols. Indeed, given a STS $\EuScript{S}_i^{N}$, all the STS
paths of $\EuScript{S}_i^{N}$, which have the same sequence of
symbols labeled on the transitions, are packed into one STS path
$b$ in $R(\EuScript{S}_i^{N})$ whose transition guards are stored
into a matrix $M_{[b]}$.

\begin{example}
    Given our trace example $t = 17011(nsys:=1, nsec:=8,
    point:=1, pid:=1)\text{ }17021(nsys:=1, nsec:=8, point:=4,
    tpoint:=9, pid:=1)$, and the reduced model depicted in Figure
    \ref{fig:sts-reduced-ch4}, $t$ is a pass trace with respect
    to $\leq_{mct}$ because each assignment $\alpha_j(x)$
    satisfies at least one guard of the matrix line $j$. For
    instance, the assignment $(point := 4)$, which is given with
    the second valued action of $t$, satisfies one of the guards
    of the second line of the matrix $M_{[b]}$.

    \begin{figure}[h]
        \begin{center}
            \includegraphics[width=1.0\linewidth]{figures/reduced_sts_ch4.png}
        \end{center}

        \caption{Reduced Symbolic Transition System model (with
        its matrix) obtained from the model depicted in Figure
        \ref{fig:sts-ch4}.}
        \label{fig:sts-reduced-ch4}
    \end{figure}
\end{example}

Given a trace $a_1(\alpha_1) \dots a_m(\alpha_m) \in
CTraces({Sut})$ and a STS path $b$ of $R(\EuScript{S}_i^{N})$
having the same sequence of symbols $a_1 \dots a_m$, the relation
can now be formulated as follows: for every valued action
$a_j(\alpha_j)$, each variable assignment $\alpha_j(x)$ must
satisfies at least one of the guards of the matrix line $j$ in
$M_{[b]}[j,*]$.

Consequently, we propose to rewrite the implementation relation
$\leq_{mct}$ below:

\begin{proposition}
    ${Sut} \leq_{mct} \EuScript{S}^{N}$ if and only if
    $\forall ~t = a_1(\alpha_1) \dots a_m(\alpha_m) \in
    CTraces({Sut})$, $\exists ~R(\EuScript{S}_i^{N}) \in
    R(\EuScript{S}^{N)}$ and
    $b = l0_{R(\EuScript{S}_i^{N})}
    \xRightarrow{(a_1(p_1),M_{[b]}[1,c_{[b]}]) \dots (a_j(p_j),M_{[b]}[j,c_{[b]}])}
    l_m$
    with $(1 \leq c_{[b]} \leq k)$ such that $\forall
    ~\alpha_j(x)_{(1 \leq j \leq m)}, \alpha_j(x) \models
    M_{[b]}[j,1] \vee \dots \vee  M_{[b]}[j,k]$, and $l_m \in Pass$.
\end{proposition}

The disjunction of guards $M_{[b]}[j,1] \vee \dots \vee
M_{[b]}[j,k]$, found in the matrix $M_{[b]}$, could be simplified
by gathering all the equalities $(x == val)$ together with
disjunctions for every variable $x$ that belongs to the parameter
set $p_j$. Such equalities can be extracted with the projection
operator \textit{proj} (see Definition \ref{def:sts}). We obtain
one guard of the form $\displaystyle \bigwedge_{x \in p_j} ((x ==
val_1) \vee \dots \vee (x == val_k))$. The STS
$D(\EuScript{S}_i^{N})$, derived from $R(\EuScript{S}_i^{N})$, is
constructed with this simplification of guards:


\begin{definition}[Derived STS $D(\EuScript{S}_i^{N})$]
    Let
    $R(\EuScript{S}_i^{N})=<L_R,l0_R,V_R,V0_R,I_R,\Lambda_R,\rightarrow_R>$
    be a STS of $R(\EuScript{S}^{N})$. We denote by
    $D(\EuScript{S}_i^{N})$ the STS $
    <L_D,l0_D,V_D,V0_D,I_D,\Lambda_D,\rightarrow_D>$ derived from
    $R(\EuScript{S}_i^{N})$ such that:

    \begin{itemize}
        \item $L_D=L_{R}$;

        \item $l0_D=l0_{R}$;

        \item $I_D=I_{R}$;

        \item $\Lambda_D=\Lambda_{R}$;

        \item $V_D, V0_D$ and $\rightarrow_D$ are defined by the
            following inference rule:

            $\frac{
                \begin{matrix}
                b=l0_{R}
                \xRightarrow{(a_1(p_1),M_{[b]}[1,c_{[b]}])\dots
                (a_m(p_m),M_{[b]}[m,c_{[b]}])}_{\rightarrow_{R}}
                l_{m},
                (1 \leq c_{[b]} \leq k) \text{ in } V0_{R}
                \end{matrix}
            }
            {
                \begin{matrix}
                l0_D
                \xRightarrow{(a_1(p_1),M_{b}[1])\dots (a_m(p_m),M_{b}[m])}_{\rightarrow_D}
                l_m, V_D=V0_D \wedge M_{b},\\
                M_{b}[j]_{(1\leq j \leq m)} = \displaystyle
                \bigwedge_{x \in p_j} ( proj_x(M_{[b]}[j, 1])
                \vee \dots \vee proj_x(M_{[b]}[j, k]))

                \end{matrix}
            }$
    \end{itemize}

    We also have $D(\EuScript{S}^{N}) =
    \{D(\EuScript{S}_1^{N}),\dots,D(\EuScript{S}_n^{N}) \}$.
\end{definition}

The second implementation relation $\leq_{mct}$ can now be
expressed by:

\begin{proposition}
    ${Sut} \leq_{mct} \EuScript{S}^{N}$ if and only if $\forall
    ~t= a_1(\alpha_1) \dots a_m(\alpha_m) \in CTraces({Sut})$,
    $\exists ~D(\EuScript{S}_i^{N}) \in D(\EuScript{S}^{N})$ and
    $l0_{D(\EuScript{S}_i^{N})} \xRightarrow{(a_1(p_1),G_{1})
    \dots (a_m(p_m),G_{m})} l_m$ such that $\forall
    ~\alpha_{j ~(1 \leq j \leq m)}, \alpha_j \models G_{j}$ and
    $l_m \in Pass$.
\end{proposition}


The implementation relation $\leq_{mct}$ now means that a trace
of $\mathit{Sut}$ must also be a pass trace of the model
$D(\EuScript{S}^{N})=\{D(\EuScript{S}_1^{N}),\dots,D(\EuScript{S}_n^{N})\}$.
Furthermore, this notion of trace inclusion can be formulated
with the first implementation relation $\leq_{ct}$ as follows:

\begin{proposition}
    ${Sut} \leq_{mct} \EuScript{S}^{N} \text{ if and only if }
    CTraces({Sut})\subseteq
    Traces_{Pass}(D(\EuScript{S}^{N}))$, and ${Sut} \leq_{mct}
    \EuScript{S}^{N} \Leftrightarrow {Sut} \leq_{ct}
    D(\EuScript{S}^{N})$.

    \label{rel:impl2}
\end{proposition}

The implementation relation $\leq_{mct}$ is now expressed with
the first relation $\leq_{ct}$, which implies that our passive
testing algorithms shall be the same for both relations except
that they shall take different reference models. In the next
section, we introduce an offline passive testing algorithm that
uses these two implementation relations.
