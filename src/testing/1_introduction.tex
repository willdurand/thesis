\section{Introduction}
\label{sec:testing:intro}

Manual testing is, by far, the most popular technique for
testing, but this technique is known to be error-prone as well.
As already formulated in this thesis, production systems are
usually composed of thousands of states (\emph{i.e.} sets of
conditions that exist at a given instant in time) and production
events, which makes testing time consuming.  In this context, we
propose a \emph{passive testing framework} for production systems
that is compound of: (i) a model inference engine, already
presented in Chapter \ref{sec:modelinf:prodsystems}, and (ii) a
passive test engine, which is the purpose of this chapter. Both
parts have to be fast and scalable to be used in practice.

The main idea of our proposal is that, given a running production
system, for which active testing cannot be applied, we extract
knowledge and models. Such models describe the functional
behaviors of the system, and may serve different purposes,
\emph{e.g.}, testing another production system. The latter can be
a new system roughly comparable to the first one in terms of
features, but it can also be an updated version of the first one.
Indeed, upgrades might inadvertently introduce faults, and it
could lead to severe damages.

In our context, testing the updated system means detecting potential
regressions before deploying changes in production. This is
usually performed by engineers at Michelin, yet their "testing"
phase is actually a manual task performed in a simulation room
where they check a couple of known scenarios for a long period
(about 6 months). Most of the time, such a process works well,
but it takes a lot of time, and there is no guarantee regarding
the number of covered scenarios. In addition, engineers who write
or maintain the applications are likely the same who perform
these manual testing phases, which can be problematic because
they often know too well the applications. We designed our
testing framework to help Michelin engineers focus on possible
failures, automatically highlighted by the framework in a short
amount of time. At the end, engineers still have to perform a few
manual tasks, but they are more efficient as they have only a few
behaviors to check (instead of all possible behaviors), given in
a reliable and fast manner.

Generally speaking, a \textit{passive tester} (also known as
observer) aims at checking whether a system under test
\emph{conforms to} a model. It can be performed in either online
or offline mode, as defined below:

\begin{itemize}
    \item \textbf{Online testing:} it means that traces are
        computed and analyzed on-the-fly to provide verdicts, and
        no trace set is studied \emph{a posteriori};

    \item \textbf{Offline testing:} it means that a set of traces
        has been collected while the system is running. Then, the
        tester gives verdicts afterwards.
\end{itemize}

In this Chapter, we present an offline passive testing technique.
We collect the traces of the system under test by reusing
\textit{Autofunk v3}'s Models generator, and we build a set of
traces whose level of abstraction is the same as those considered
for inferring models. Then, we use these traces to check whether
the system under test conforms to the inferred models.  In
previous works, we used to work with fixed sets of traces. We
noticed that, by taking large trace sets, we could build more
complete models, and performing offline passive testing allows to
use such large trace sets.

\emph{Conformance} is defined with two implementation relations,
which express precisely what the system under test should do. The
first relation is based on the \emph{trace preorder}
\cite{DNH84}, which is a well-known relation based upon trace
inclusion, and heavily used with passive testing.  Nevertheless,
our inferred models are partials, \emph{i.e.} they do not
necessarily capture all the possible behaviors that should
happen. That is why we propose a second implementation relation,
less restrictive on the traces that should be observed from the
system under test.

In Section \ref{sec:testing:normal}, we introduce an extra step
of the model inference method described in Chapter
\ref{sec:modelinf:prodsystems} that is required to enable
testing. In Section \ref{sec:testing:passive}, we present our
offline passive testing technique, built on-top of this model
inference framework. We present key results on offline passive
testing in Section \ref{sec:testing:offline:impl-exp}. Finally,
we conclude on this chapter in Section
\ref{sec:testing:conclusion}.

\textbf{Publication.} This work has been partially published in
the Proceedings of the 13th International Conference on Formal
Methods and Models for Co-Design (MEMOCODE'15) \cite{7340480}.

%%%%%%%%%%%%%%%%%%%%%%%%%%%%%%%%%%%%%%%%%%%%%%%%%%%%%%%%%%%%%%%%%

\section{Normalization of the inferred models}
\label{sec:testing:normal}

In order to perform testing, we reuse the reduced model set
$R(\EuScript{S}) = \{R(\EuScript{S}_1),\dots,R(\EuScript{S}_n)\\
\}$ inferred with \textit{Autofunk} that we \textit{normalize} to
get rid of some runtime-dependent information of the system under
analysis $\mathit{Sua}$. Indeed, both models $\EuScript{S}_i$ and
$R(\EuScript{S}_i)$ include parameters that are dependent to the
products being manufactured. That is a consequence of generating
models that describe behaviors of a continuous stream of products
that are strictly identified. For instance, each action in a
given sequence owns the assignment $(pid := val)$ (for the
record, $pid$ stands for \emph{product identifier}). Collected
traces are also time-dependent, which would make testing of
another production system unfeasible. These information are
typically known by a human domain expert, and we can use
inference rules to identify and remove assignments one more time.

Given the model sets $\EuScript{S} =
\{\EuScript{S}_1,\dots,\EuScript{S}_n\}$ and $R(\EuScript{S}) =
\{R(\EuScript{S}_1),\dots,R(\EuScript{S}_n)\}$, we remove the
assignments related to product identifiers and time stamps to
produce normalized model sets, denoted by $\EuScript{S}^{N} = \{
\EuScript{S}_1^{N}, \dots, \EuScript{S}_n^{N} \}$ and
$R(\EuScript{S}^{N}) = \{ R(\EuScript{S}_1^{N}), \dots,
R(\EuScript{S}_n^{N}) \}$.

Furthermore, we label all the final locations with
"Pass". We flag these locations as \emph{verdict locations}, and
gather them in the set $Pass \subseteq L_{\EuScript{S}_i^{N}}$.
Both $\EuScript{S}_i^{N}$ and $R(\EuScript{S}_i^{N})$ represent more
generic models, \emph{i.e.}  they express \textit{some possible
complete behaviors that should happen}. These behaviors are
represented by the traces $Traces_{Pass}
(\EuScript{S}^{N})=\displaystyle \bigcup_{1 \leq i \leq n}
Traces_{Pass} (\EuScript{S}_i^{N})=Traces_{Pass}
(R(\EuScript{S}^{N}))$. We refer to these traces as \textit{pass
traces}, and we call the other traces \textit{possibly fail
traces}.

%%%%%%%%%%%%%%%%%%%%%%%%%%%%%%%%%%%%%%%%%%%%%%%%%%%%%%%%%%%%%%%%%

\section{Passive testing with \textit{Autofunk}}
\label{sec:testing:passive}

We consider both model sets $\EuScript{S}^{N}$ and
$R(\EuScript{S}^{N})$ of a system under analysis $\mathit{Sua}$,
generated by our inference-based model generation framework, as
\emph{reference models}. In this section, we present the second
part of our testing framework, dedicated to the passive testing
of a system under test $\mathit{Sut}$.

Figure \ref{fig:passive-autofunk} depicts the design of our
offline passive testing technique. A set of production events has
been collected beforehand from $\mathit{Sut}$ in the same way as
for $\mathit{Sua}$. These events are grouped into traces to form
the trace set $Traces({Sut})$, and then filtered to obtain a set
of complete traces denoted by $CTraces({Sut})$ (cf.
\crossref{sec:modelinf:prodsystems}{sec:modelinf:prodsystems:better-segmentation}).
Finally, we perform passive testing to \emph{check} if
$\mathit{Sut}$ conforms to $\EuScript{S}^{N}$.

\begin{figure}[h]
    \begin{center}
        \includegraphics[width=1.0\linewidth]{figures/passive_autofunk.png}
    \end{center}

    \caption{Overview of \textit{Autofunk v3} with the
    passive testing extension. While the previous
    \textit{Autofunk} design has been kept, there are
    two new modules: "STS Normalization" and "check",
    representing the passive conformance testing part.}
    \label{fig:passive-autofunk}
\end{figure}

Our industrial partner wishes to check whether every complete
execution trace of $\mathit{Sut}$ matches a behavior captured by
$\EuScript{S}^{N}$. In this case, the test verdict must reflect a
successful result. On the contrary, if a complete execution of
$\mathit{Sut}$ is not captured by $\EuScript{S}^{N}$, one cannot
conclude that $\mathit{Sut}$ is faulty because $\EuScript{S}^{N}$
is a partial model, and it does not necessarily includes all the
correct behaviors. Below, we formalize these verdict notions
with two implementation relations. Such relations between models
can only be written by assuming the following \emph{test
assumption}: the black-box system $\mathit{Sut}$ can be described
by a model, here with a Labeled Transition System as defined in
\crossref{sec:related:testing}{sec:definitions:lts}. For
simplicity purpose, we also denote this model by $\mathit{Sut}$.

It is manifest that both the models and $\mathit{Sut}$ must be
\emph{compatible}, \emph{i.e.} the production events captured on
$\mathit{Sua}$ and $\mathit{Sut}$ must lead to similar valued
events. Otherwise, all traces of $\mathit{Sut}$ will likely be
rejected, and the testing process will become wasteful:

\begin{definition}[Compatibility of $\EuScript{S}^N$ and $\mathit{Sut}$]
    Let $\EuScript{S}^N = \{ \EuScript{S}_1^N, \dots,
    \EuScript{S}_n^N \}$ be a STS set, and $\mathit{Sut}$ be a LTS
    (semantics), which is assumed to behave as the production
    system.

    $\EuScript{S}^N$ and $\mathit{Sut}$ are said \emph{compatible}
    if and only if: $\displaystyle\Sigma_{Sut} \subseteq \bigcup_{1 \leq i \leq n}
    \bigcup_{a(p) \in \Lambda_{\EuScript{S}_i^N}} a \times D_p$.

    \label{def:similar-sut-sua}
\end{definition}

\subsection{First implementation relation: $\leq_{ct}$}

The first implementation relation, denoted by $\leq_{ct}$, refers
to the trace preorder relation
\cite{DNH84,vaandrager1991relationship} (cf. Example
\ref{example:trace_preorder} on page
\pageref{example:trace_preorder}).
It aims at checking whether all the complete execution traces of
$\mathit{Sut}$ are pass traces of
$\EuScript{S}^{N}=\{\EuScript{S}_1^{N},\dots,\EuScript{S}_n^{N}\}$.
The first implementation relation can be written with the
following definition:

\begin{definition}[Implementation relation $\leq_{ct}$]
\label{rel:impl1}

Let $\EuScript{S}^{N}$ be an inferred model of $\mathit{Sua}$, and
$\mathit{Sut}$ be the system under test. When $\mathit{Sut}$
produces complete traces also captured by $\EuScript{S}^{N}$, we
write: ${Sut} \leq_{ct} \EuScript{S}^{N} =_{def} CTraces({Sut})
\subseteq  Traces_{Pass}(\EuScript{S}^{N})$.
\end{definition}

Pragmatically, the reduced model set $R(\EuScript{S}^{N})$ sounds
more convenient for passively testing $\mathit{Sut}$ because it
is strongly reduced in terms of size compared to
$\EuScript{S}^{N}$. The test relation can also be written
as below because both models $\EuScript{S}^{N}$ and
$R(\EuScript{S}^{N})$ are trace equivalent (cf. Proposition
\ref{prop:trace-eq-r} in Chapter \ref{sec:modelinf:prodsystems}):

\begin{proposition}
\label{rel:impl12}
${Sut} \leq_{ct} \EuScript{S}^{N} \Leftrightarrow CTraces({Sut})
\subseteq  Traces_{Pass}(R(\EuScript{S}^{N}))$.
\end{proposition}

\subsection{Second implementation relation: $\leq_{mct}$}

As stated previously, the inferred model $\EuScript{S}^{N}$ of
$\mathit{Sua}$ is partial, and it might not capture all the
behaviors that should happen on $\mathit{Sut}$. Consequently,
our partner wants a \emph{weaker implementation relation} that is
less restrictive on the traces that should be observed from
$\mathit{Sut}$. This second relation aims to
check that, for every complete trace $t=(a_1(p), \alpha_1) \dots
(a_m(p), \alpha_m)$ of $\mathit{Sut}$, we also have a set of traces of
$Traces_{Pass}(\EuScript{S}^{N})$ having the same sequence of
symbols such that every variable assignment $\alpha_j(x)_{(1 \leq
j \leq m)}$ of $t$ is found in one of the traces of
$Traces_{Pass}(\EuScript{S}^{N})$ with the same symbol $a_j$.

\begin{example}
    Figure \ref{fig:sts-ch4} recalls the example considered in
    Chapter \ref{sec:modelinf:prodsystems}. The trace $t =
    (17011(\{ nsys, nsec, point, pid \}), \{
    nsys:=1, nsec:=8, point:=1, pid:=1 \})\text{ }
    17021(\{ nsys, nsec, point, tpoint, pid \}, \{
    nsys:=1, nsec:=8, point:=4, tpoint\\:=9, pid:=1 \})$ is not
    a \emph{pass trace} of $\EuScript{S}^{N}$ because this trace
    cannot be extracted from one of the paths of the STS depicted
    in Figure \ref{fig:sts-ch4}, on account of the variables
    $point$ and $tpoint$, which do not take the expected values.
    Nonetheless, both variables are assigned with $(point := 4)$
    and $(tpoint := 9)$ in the second path. This is interesting
    as it indicates that such values \emph{may} be correct since
    they are actually used in a similar action in a similar path.
    Our second implementation relation aims at expressing that
    such a trace $t$ captures a correct behavior as well.

    \begin{figure}[ht]
        \begin{center}
            \includegraphics[width=1.0\linewidth]{figures/STS1.png}
        \end{center}

        \caption{The first Symbolic Transition System inferred in
        Chapter \ref{sec:modelinf:prodsystems}.}
        \label{fig:sts-ch4}
    \end{figure}
\end{example}

The second implementation relation, denoted by $\leq_{mct}$, is
defined as follows:

\begin{definition}[Implementation relation $\leq_{mct}$]
     Let $\EuScript{S}^{N}$ be an inferred model of
     $\mathit{Sua}$, and $\mathit{Sut}$ be a system under test.
     We write: ${Sut} \leq_{mct} \EuScript{S}^{N}$ if and only if
     $\forall ~t= (a_1(p), \alpha_1) \dots (a_m(p), \alpha_m) \in
     CTraces({Sut})$, $\forall ~\alpha_j(x)_{(1 \leq j \leq m)},
     \exists ~\EuScript{S}_i^{N} \in \EuScript{S}^{N}$ and $t'
     \in Traces_{Pass}(\EuScript{S}_i^{N})$ such that
     $t'=(a_1(p), \alpha_1') \dots (a_m(p), \alpha_m')$ and
     $\alpha_j'(x)=\alpha_j(x)$.

     \label{impl21}
\end{definition}

According to the above definition, the successive symbols and
variable assignments of a trace $t \in CTraces({Sut})$ can be
found into several traces of $Traces_{Pass}(\EuScript{S}_i^{N})$,
which have the same sequence of symbols $a_1 \dots a_m$ as the
trace $t$. The reduced model $R(\EuScript{S}_i^{N})$ was
previously constructed to capture all these traces in
$Traces_{Pass}(\EuScript{S}_i^{N})$, having the same sequence of
symbols. Indeed, given a STS $\EuScript{S}_i^{N}$, all the STS
paths of $\EuScript{S}_i^{N}$, which have the same sequence of
symbols labeled on the transitions, are packed into one STS path
$b$ in $R(\EuScript{S}_i^{N})$ whose transition guards are stored
into a matrix $M_{[b]}$.

\begin{example}
    Given our trace example $t = (17011(\{ nsys, nsec, point, pid
    \}), \{ nsys\\:=1, nsec:=8, point:=1, pid:=1 \})\text{ }
    17021(\{ nsys, nsec, point, tpoint, pid \}, \{ nsys\\:=1,
    nsec:=8, point:=4, tpoint:=9, pid:=1 \})$, and the reduced
    model depicted in Figure \ref{fig:sts-reduced-ch4}, $t$ is a
    pass trace with respect to $\leq_{mct}$ because each
    assignment $\alpha_j(x)$ satisfies at least one guard of the
    matrix line $j$. For instance, the assignment $(point := 4)$,
    which is given with the second valued event of $t$,
    satisfies one of the guards of the second line of the matrix
    $M_{[b]}$.

    \begin{figure}[h]
        \begin{center}
            \includegraphics[width=1.0\linewidth]{figures/reduced_sts_ch4.png}
        \end{center}

        \caption{Reduced Symbolic Transition System model (with
        its matrix) obtained from the model depicted in Figure
        \ref{fig:sts-ch4}.}
        \label{fig:sts-reduced-ch4}
    \end{figure}
\end{example}

Given a trace $(a_1(p), \alpha_1) \dots (a_m(p), \alpha_m) \in
CTraces({Sut})$ and a STS path $b$ of $R(\EuScript{S}_i^{N})$
having the same sequence of symbols $a_1 \dots a_m$, the relation
can now be formulated as follows: for every valued event
$(a_j(p), \alpha_j)$, each variable assignment $\alpha_j(x)$ must
satisfies at least one of the guards of the matrix line $j$ in
$M_{[b]}[j,*]$.

Consequently, we propose to rewrite the implementation relation
$\leq_{mct}$ as:

\begin{proposition}
    ${Sut} \leq_{mct} \EuScript{S}^{N}$ if and only if
    $\forall ~t = (a_1(p), \alpha_1) \dots (a_m(p), \alpha_m) \in
    CTraces({Sut})$, $\exists ~R(\EuScript{S}_i^{N}) \in
    R(\EuScript{S}^{N})$ and
    $b = l0_{R(\EuScript{S}_i^{N})}
    \xRightarrow{(a_1(p_1),M_{[b]}[1,c_{[b]}]) \dots (a_j(p_j),M_{[b]}[j,c_{[b]}])}
    l_m$
    with $(1 \leq c_{[b]} \leq k)$ such that $\forall
    ~\alpha_j(x)_{(1 \leq j \leq m)}, \alpha_j(x) \models
    M_{[b]}[j,1] \vee \dots \vee  M_{[b]}[j,k]$, and $l_m \in Pass$.
\end{proposition}

The disjunction of guards $M_{[b]}[j,1] \vee \dots \vee
M_{[b]}[j,k]$, found in the matrix $M_{[b]}$, could be simplified
by gathering all the equalities $(x == val)$ together with
disjunctions for every variable $x$ that belongs to the parameter
set $p_j$ of $a_j(p_j)$. Such equalities can be extracted with
the projection operator \textit{proj} (see Definition
\ref{def:sts}). We obtain one guard of the form $\displaystyle
\bigwedge_{x \in p_j} ((x == val_1) \vee \dots \vee (x ==
val_k))$. This can be expressed by deriving new STSs from
$R(\EuScript{S})$.

\subsubsection{STS $D(\EuScript{S})$}

Given a STS $R(\EuScript{S}_i^{N})$, the STS
$D(\EuScript{S}_i^{N})$ is constructed with the disjunction of
guards described in the previous section:

\begin{definition}[Derived STS $D(\EuScript{S}_i^{N})$]
    Let
    $R(\EuScript{S}_i^{N})=<L_R,l0_R,V_R,V0_R,I_R,\Lambda_R,\rightarrow_R>$
    be a STS of $R(\EuScript{S}^{N})$. We denote by
    $D(\EuScript{S}_i^{N})$ the STS $
    <L_D,l0_D,V_D,V0_D,I_D,\Lambda_D,\rightarrow_D>$ derived from
    $R(\EuScript{S}_i^{N})$ such that:

    \begin{itemize}
        \item $L_D=L_{R}$;

        \item $l0_D=l0_{R}$;

        \item $I_D=I_{R}$;

        \item $\Lambda_D=\Lambda_{R}$;

        \item $V_D, V0_D$ and $\rightarrow_D$ are defined by the
            following inference rule:

            $\frac{
                \begin{matrix}
                b=l0_{R}
                \xRightarrow{(a_1(p_1),M_{[b]}[1,c_{[b]}])\dots
                (a_m(p_m),M_{[b]}[m,c_{[b]}])}_{R}
                l_{m},
                (1 \leq c_{[b]} \leq k) \text{ in } V0_{R}
                \end{matrix}
            }
            {
                \begin{matrix}
                l0_D
                \xRightarrow{(a_1(p_1),M_{b}[1])\dots (a_m(p_m),M_{b}[m])}_{D}
                l_m, V0_D=V0_D \wedge M_{b},\\
                M_{b}[j]_{(1\leq j \leq m)} = \displaystyle
                \bigwedge_{x \in p_j} ( proj_x(M_{[b]}[j, 1])
                \vee \dots \vee proj_x(M_{[b]}[j, k]))

                \end{matrix}
            }$
    \end{itemize}

    The set of derived models is denoted by $D(\EuScript{S}^{N}) =
    \{D(\EuScript{S}_1^{N}),\dots,D(\EuScript{S}_n^{N}) \}$.

    \label{def:sts-d}
\end{definition}

The second implementation relation $\leq_{mct}$ can now be
expressed by:

\begin{proposition}
    ${Sut} \leq_{mct} \EuScript{S}^{N}$ if and only if $\forall
    ~t= (a_1(p), \alpha_1) \dots (a_m(p), \alpha_m) \in CTraces({Sut})$,
    $\exists ~D(\EuScript{S}_i^{N}) \in D(\EuScript{S}^{N})$ and
    $l0_{D(\EuScript{S}_i^{N})} \xRightarrow{(a_1(p_1),G_{1})
    \dots (a_m(p_m),G_{m})} l_m$ such that $\forall
    ~\alpha_{j ~(1 \leq j \leq m)}, \alpha_j \models G_{j}$ and
    $l_m \in Pass$.
\end{proposition}

The implementation relation $\leq_{mct}$ now means that a trace
of $\mathit{Sut}$ must also be a pass trace of the model set
$D(\EuScript{S}^{N})=\{D(\EuScript{S}_1^{N}),\dots,D(\EuScript{S}_n^{N})\}$.
This notion of trace inclusion can also be formulated with the
first implementation relation $\leq_{ct}$ as follows:

\begin{proposition}
    ${Sut} \leq_{mct} \EuScript{S}^{N} \text{ if and only if }
    CTraces({Sut})\subseteq
    Traces_{Pass}(D(\EuScript{S}^{N}))$, and ${Sut} \leq_{mct}
    \EuScript{S}^{N} \Leftrightarrow {Sut} \leq_{ct}
    D(\EuScript{S}^{N})$.

    \label{rel:impl2}
\end{proposition}

Now, the implementation relation $\leq_{mct}$ is expressed with
the first relation $\leq_{ct}$, which implies that our passive
testing algorithms shall be the same for both relations except
that they shall take different reference models.

Furthermore, because $D(\EuScript{S}^{N})$ is derived from
$R(\EuScript{S}^{N})$, \emph{i.e.} the guards of the model
$D(\EuScript{S}_i^{N})$ are the disjunctions of the guards of the
model $R(\EuScript{S}_i^{N})$ (cf. Definition \ref{def:sts-d}),
we have $Traces_{Pass}(R(\EuScript{S}^{N})) \subseteq
Traces_{Pass}(D(\EuScript{S}^{N}))$. Therefore, if $\mathit{Sut}
\leq_{ct} \EuScript{S}^N$, we have $\mathit{Sut} \leq_{mct}
\EuScript{S}^N$:

\begin{proposition}
    $\mathit{Sut} \leq_{ct} \EuScript{S}^N \implies
    \mathit{Sut} \leq_{mct} \EuScript{S}^N$.

    \label{rel:impl-ct-implies-mct}
\end{proposition}

In the next section, we introduce the offline passive testing
algorithm that uses these two implementation relations.
