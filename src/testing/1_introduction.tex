\section{Introduction}
\label{sec:testing:intro}

Manual testing is, by far, the most popular technique for
testing, but this technique is known to be error-prone as well.
Additionally, production systems are usually composed of
thousands of states (i.e. sets of conditions that exist at a
given instant in time) and production events, which makes
testing time consuming. Our industrial partner Michelin is a
worldwide tire manufacturer that designs most of its factories,
production systems, and software by itself. In a factory, there
are different workshops for each step of the tire building
process. At a workshop level, we observe a continuous stream of
products from specific entry points (railroad switches) to a
finite set of exit points, constituting production lines.
Thousands of \emph{production events} are exchanged among the
industrial devices of the same workshop every day, allowing some
factories to build over 30,000 tires a day.

In this context, we propose a testing framework for production
systems that is composed of two parts: a model inference engine,
as already presented in Chapter \ref{sec:modelinf:prodsystems},
and a passive testing engine that is presented in this chapter.
Both parts have to be fast and scalable to be used in practice.
The main idea of our proposal is that, given a running production
system, we extract knowledge and models by passively monitoring
it. Such models describe the functional behaviours of the system,
and may serve for different purposes, e.g., testing of a second
production system. The latter can be a new system roughly
comparable to the first one in terms of features, but it can also
be an updated version of the first one. Indeed, upgrades might
inadvertently introduce or create faults, and could lead to
severe damages. Here, testing the updated system means detecting
potential regressions before deploying changes in production.

A \textit{passive tester} (a.k.a. observer) aims at checking
whether a system under test conforms to an inferred model. It can
be performed in either online or offline mode, as defined below:

\begin{itemize}
\item \textbf{Online testing:} sometimes called on-the-fly
testing, it is a technique in which test case generation and test
execution are combined into a single algorithm. The tester gives
a verdict everytime a trace incomes,

\item \textbf{Offline testing:} it means that a set of traces has
been collected while the system is running. Then, the tester
gives verdicts.
\end{itemize}

We collect the traces of the system under test by reusing
Autofunk's Models generator. In offline mode, we build a set of
traces with the same level of abstraction as those considered for
inferring models. In online mode, we apply the same process for
each new incoming trace. Then, we use these traces to check if
the system under test conforms to the inferred models.
Conformance is defined with two implementation relations, which
express precisely what the system under test should do. The first
relation corresponds to the trace preoder \cite{DNH84}, which is
a well-known relation based upon trace inclusion, and heavily
used with passive testing.  Nevertheless, our inferred models are
partials, i.e.  they do not necessarily capture all the possible
behaviours that should happen. That is why we propose a second
implementation relation, less restrictive on the traces that
should be observed from the system under test.

Both offline and online modes are not completely unalike, they
also serve different purposes. In previous works, we used to work
with fixed sets of traces. We noticed that, by taking large sets,
we could build more complete models. Performing offline passive
testing allows to use such large trace sets, hence the intuition
that we should be able to validate more behaviors. Our online
passive testing approach records traces on a system under test on
the fly, and then check whether those traces satisfy
specifications still generated from a system under analysis. It
enables what we call just-in-time fault detection. Faults can be
revealed in near real-time on a running system so that users can
be notified as soon as possible.

First of all, we present a better solution to segment and filter
the initial trace set given as input in Section
\ref{sec:testing:segmentation}. Indeed, with the statistical
analysis introduced in
\crossref{sec:modelinf:prodsystems}{sec:modelinf:prodsystems:segmentation},
we hit several problems, mainly because our algorithm was not
stable, in other words the use of a configurable minimum limit
was not accurate enough. We simply could not rely on it for
testing purpose.  In Section \ref{sec:testing:normal}, we
introduce a new step of the model inference method described in
Chapter \ref{sec:modelinf:prodsystems} that is required to enable
testing. In Section \ref{sec:testing:passive}, we present both
our offline and online passive testing techniques on-top of this
model inference framework respectively in Section
\ref{sec:testing:offline} and Section \ref{sec:testing:online}.
We present key results on offline passive testing in Section
\ref{sec:testing:offline:impl-exp}.  Finally, we conclude on this
work in Section \ref{sec:testing:conclusion}.

\textbf{Publication.} This work has been partially published in
the Proceedings of the 13th International Conference on Formal
Methods and Models for Co-Design (MEMOCODE'15).

%%%%%%%%%%%%%%%%%%%%%%%%%%%%%%%%%%%%%%%%%%%%%%%%%%%%%%%%%%%%%%%%%

\section{A better solution to trace segmentation and filtering}
\label{sec:testing:segmentation}

In
\crossref{sec:modelinf:prodsystems}{sec:modelinf:prodsystems:segmentation},
we presented a statistical analysis to segment and filter the
initial trace set used to infer models. However, this method was
not stable, and we decided to rework it. We now rely on a machine
learning technique to segment it into several subsets, one per
entry point of the system $\mathit{Sua}$. We leverage this
process to also remove incomplete traces, i.e. traces that do not
express an execution starting from an entry point and ending to
an exit point. These can be extracted by analysing the traces and
the variable $point$, which captures the product physical
location.

\begin{figure}[ht]
    \includegraphics[width=1.0\linewidth]{figures/kmeans.png}

    \caption{k-means clustering explained}
    \label{fig:kmeans}
\end{figure}

In order to determine both entry and exit points of
$\mathit{Sua}$, we rely on an outlier detection approach
\cite{1695852}. An outlier is an observation that deviates so
much from the other observations as to arouse suspicions that it
was generated by a different mechanism. More precisely, we chose
to use the \textit{k-means clustering} method
\cite{10.2307/2346830}, a machine learning algorithm, which is
both fast and efficient, and does not need to be trained before
being effectively used (that is called unsupervised learning, and
it is well-known in the machine learning field). \textit{k-means
clustering} aims to partition $n$ observations into $k$ clusters
as shown in Figure \ref{fig:kmeans}.

In our context, observations are represented by the variable
$point$ present in each trace of $Traces({Sua})$, which captures
the product physical location, and $k=2$ as we want to group the
outliers together, and leave the other points in another cluster.
But, since we want to leverage the largest part of the initial
trace set, we apply \textit{k-means clustering} several times
until reaching 80\% of traces retained.

Once the entry and exit points are found, we segment
$Traces({Sua})$ and obtain a set $CTraces({Sua})=\{ST_1, \dots,
ST_n\}$. Then, we apply the same generation and reduction steps
as described in
\crossref{sec:modelinf:prodsystems}{sec:modelinf:prodsystems:generation}
and
\crossref{sec:modelinf:prodsystems}{sec:modelinf:prodsystems:reduction}
so that we obtain the reduced model $R(\EuScript{S}) =
\{R(\EuScript{S}_1),\dots,R(\EuScript{S}_n)\}$.

%%%%%%%%%%%%%%%%%%%%%%%%%%%%%%%%%%%%%%%%%%%%%%%%%%%%%%%%%%%%%%%%%

\section{Extending inferred models with normalisation}
\label{sec:testing:normal}

In order to perform testing, we reuse the reduced model
$R(\EuScript{S}) = \{R(\EuScript{S}_1),\dots,R(\EuScript{S}_n)\}$
inferred with Autofunk that we \textit{normalise} to get rid of
some runtime-dependent information.  Indeed, both models
$\EuScript{S}$ and $R(\EuScript{S})$ include parameters that are
dependent to the products being manufactured.  That is a
consequence of generating models that describe behaviours of a
continuous stream of products which are strictly identified, i.e.
for each action in a given sequence, we have the assignment $(pid
= val)$ ($pid$ stands for product identifier).  Here, we
normalise these models before using them for testing.  The
resulting models are denoted with $\EuScript{S}^{N}$ and
$R(\EuScript{S}^{N})$.

We remove the assignments relative to product identifiers and
timestamps. Furthermore, we label all the final locations with
"Pass". We denote these locations as verdict locations and gather
them in the set $Pass \subseteq L_{\EuScript{S_i}^{N}}$. Both
$\EuScript{S}^{N}$ and $R(\EuScript{S}^{N})$ represent more
generic models, i.e.  they express \textit{some possible
behaviours that should happen}. These behaviours are represented
by the traces $Traces_{Pass} (\EuScript{S}^{N})=\displaystyle
\bigcup_{1 \leq i \leq n} Traces_{Pass}
(\EuScript{S}_i^{N})=Traces_{Pass} (R(\EuScript{S}^{N}))$. We
refer to these traces as \textit{pass traces}. We call the other
traces \textit{possibly fail traces}.

%%%%%%%%%%%%%%%%%%%%%%%%%%%%%%%%%%%%%%%%%%%%%%%%%%%%%%%%%%%%%%%%%

\section{Passive testing with Autofunk}
\label{sec:testing:passive}

We consider both models $\EuScript{S}^{N}$ and
$R(\EuScript{S}^{N})$ of a system under analysis $\mathit{Sua}$,
generated by our inference-based model generation framework, as
reference models. In this section, we present the second part of
our framework, dedicated to the passive testing of a system under
test $\mathit{Sut}$.

\begin{figure}[ht]
\includegraphics[width=0.85\linewidth]{figures/passive_autofunk.png}

\caption{Autofunk's passive testing architecture}
\label{fig:passive-autofunk}
\end{figure}

Figure \ref{fig:passive-autofunk} depicts the architecture of our
passive testing technique. In offline mode, a set of production
events has been collected beforehand from $\mathit{Sut}$ in the
same way as for $\mathit{Sua}$. These are grouped into traces to
form the trace set $Traces({Sut})$, and then filtered as
explained in Section \ref{sec:testing:segmentation} to obtain a
set of complete traces denoted with $CTraces({Sut})$. In online
mode, we don't have traces but rather filtered valued events of
$\mathit{Sut}$ coming on-the-fly. Finally, we perform passive
testing to check if $\mathit{Sut}$ conforms to
$\EuScript{S}^{N}$.

\TODO{figure? maybe 2?}

Our industrial partner wishes to check whether every complete
execution trace of $\mathit{Sut}$ matches a behaviour captured by
$\EuScript{S}^{N}$. In this case, the test verdict must reflect a
successful result. On the contrary, if an execution of
$\mathit{Sut}$ is not captured by $\EuScript{S}^{N}$, one cannot
conclude that $\mathit{Sut}$ is faulty because $\EuScript{S}^{N}$
is a partial model, and it does not necessarily includes all the
correct behaviours. Below, we formalise theses verdict notions
with two implementation relations. Such relations between models
can only be written by assuming the following classical test
assumption: the black-box system $\mathit{Sut}$ can be described
by a model, here with a LTS as defined in
\crossref{sec:related:testing}{sec:definitions:lts}. We also
denote this model with $\mathit{Sut}$.

The first implementation relation, denoted with $\leq_{ct}$,
refers to the trace preorder relation \cite{DNH84}. It aims at
checking whether all the complete execution traces of
$\mathit{Sut}$ are pass traces of
$\EuScript{S}^{N}=\{\EuScript{S}_1^{N},\dots,\EuScript{S}_n^{N}\}$.
The first implementation relation can be written with the
following definition:

\begin{definition}[The $\leq_{ct}$ implementation relation]
\label{rel:impl1}

Let $\EuScript{S}^{N}$ be an inferred model of $\mathit{Sua}$ and
$\mathit{Sut}$ be the system under test. When $\mathit{Sut}$
produces complete traces also captured by $\EuScript{S}^{N}$, we
write: ${Sut} \leq_{ct} \EuScript{S}^{N} =_{def} CTraces({Sut})
\subseteq  Traces_{Pass}(\EuScript{S}^{N})$.
\end{definition}

Pragmatically, the reduced model $R(\EuScript{S}^{N})$ sounds
more convenient for passively testing $\mathit{Sut}$ since it is
strongly reduced in terms of size compared to $\EuScript{S}^{N}$.
The test relation can also be written as below since both models
$\EuScript{S}^{N}$ and $R(\EuScript{S}^{N})$ are trace equivalent:

\begin{proposition}
\label{rel:impl12}
${Sut} \leq_{ct} \EuScript{S}^{N} \text{ iff } CTraces({Sut})
\subseteq  Traces_{Pass}(R(\EuScript{S}^{N}))$.
\end{proposition}

As stated previously, the inferred model $\EuScript{S}^{N}$ of
$\mathit{Sua}$ is partial, and might not capture all the
behaviours that should happen on $\mathit{Sut}$. Consequently,
our partner wants a weaker implementation relation, which is less
restrictive on the traces that should be observed from
$\mathit{Sut}$.  Intuitively, this second relation aims to check
that, for every complete trace $t=a_1(\alpha_1)...a_m(\alpha_m)$
of $\mathit{Sut}$, we also have a set of traces of
$Traces_{Pass}(\EuScript{S}^{N})$ having the same sequence of
symbols such that every variable assignment $\alpha_j(x)_{(1 \leq
j \leq m)}$ of $t$ is found in one of the traces of
$Traces_{Pass}(\EuScript{S}^{N})$ with the same symbol $a_j$.
If we take back the example of Figure \ref{fig:firstmodel}, the
trace $t=(17011(nsys=1,nsec=8,point=1,pid=1)\text{ }
17021(nsys=1,nsec=8,point=4,tpoint=9,pid=1)$ is not a pass trace
of $\EuScript{S}^{N}$ because this trace cannot be extracted from
one of the paths of the STS of Figure \ref{fig:firstmodel}, on
account of the variables $point$ and $tpoint$, which do not take
the expected values. However, both variables are assigned with
$point=4,tpoint=9$ in the second path. This is interesting as it
indicates that such values may be correct since they are actually
used in a similar action in a similar path. Here, the second
implementation relation aims at expressing that this trace $t$
captures a correct behaviour as well.

% \begin{comment}
% This second implementation relation, denoted with $\leq_{mct}$, is written
% with the classes of equivalent paths $[b]$, found in a STS of
% $\EuScript{S}^{N}=\{\EuScript{S}_1^{N},...,\EuScript{S}_n^{N}\}$.
% Indeed, an equivalent class $[b]$ of $\EuScript{S}_i^{N}$ gathers
% STS paths having the same sequence of labels but with
% different guards. The definition of $\leq_{mct}$, given below,
% means that for a trace $t$, an equivalence class $[b]$ must be
% found such that for every valued action $a_j(\alpha_j)$, each
% variable assignment $\alpha_j(x)$ with $x$ a variable, must
% satisfies at least on the guards $G_{j1},...,G_{jk}$ found on
% the transitions labelled with $a_j(p_j)$:
%
% \begin{proposition}
%     Let $\mathit{Sua}$ be the system under analysis,
%     $\EuScript{S}^{N}=\{\EuScript{S}_1^{N},\dots,\EuScript{S}_n^{N}\}$ be its inferred model and $\mathit{Sut}$ be the system under test.
%     We denote ${Sut} \leq_{mct} \EuScript{S}^{N} =_{def} \forall t=
%     a_1(\alpha_1)...a_m(\alpha_m) \in CTraces({Sut}), \exists
%     \EuScript{S}_i^{N} \text{ and } [b]=\{l0_{\EuScript{S}_i^{N}} \xRightarrow{(a_1(p_1),G_{1o}),...,(a_m(p_m),G_{mo})}l_{mo} (1 \leq o \leq k) \}$ such that $\forall \alpha_j(x) (1 \leq j \leq m), \alpha_j(x) \models G_{j1} \vee ... \vee  G_{jk}$ and $l_{mo} \in Pass$.
% \end{proposition}
%
% This relation can be reformulated with the reduced model
% $R(\EuScript{S}^{N})=\{R(\EuScript{S}_1^{N}),\dots,R(\EuScript{S}_n^{N})\}$,
% which reduces the equivalence classes $[b]$ into one STS path.
% Given a trace $a_1(\alpha_1)...a_m(\alpha_m)$, the relation can
% be now written with matrices of guards: for every valued action
% $a_j(\alpha_j)$, each variable assignment $\alpha_j(x)$ must now
% satisfies one of the guards of the matrix line $j$ in
% $M_{[b]}[j,*]$.  If, we take back the trace example
% $t=(17011(nsys=1,nsec=8,point=1,pid=1)\text{ }
% 17021(nsys=1,nsec=8,point=4,tpoint=9,pid=1)$, and the STS of
% Figure \ref{fig:reduced-model}, the assignment $point=4$, which
% is given with the second valued action of $t$, satisfies one of
% the guards of the second line of the matrix $M_{[b]}$.
% \end{comment}

This implementation relation, denoted with $\leq_{mct}$, is
written with:

\begin{definition}[The $\leq_{mct}$ implementation relation]
	\label{impl21}
	 Let $\EuScript{S}^{N}$ be an inferred model of $\mathit{Sua}$ and
	 $\mathit{Sut}$ be the system under test.

     We denote ${Sut} \leq_{mct} \EuScript{S}^{N} =_{def} \forall
     t= a_1(\alpha_1) \dots a_m(\alpha_m) \in CTraces({Sut}),
     \forall \alpha_j(x)_{(1 \leq j \leq m)},\\ \exists
     \EuScript{S}_i^{N} \in \EuScript{S}^{N} \text{ and } t'\in
     Traces_{Pass}(\EuScript{S}_i^{N})$ such that
     $t'=a_1(\alpha_1')...a_m(\alpha_m')$ \text{ and }
     $\alpha_j'(x)=\alpha_j(x)$.
\end{definition}

In the following, we rewrite this relation in an equivalent but
simpler form. According to the above definition, the successive
symbols and variable assignments of a trace $t \in
CTraces({Sut})$ must be found into several traces of
$Traces_{Pass}(\EuScript{S}_i^{N})$, which have the same sequence
of symbols $a_1...a_m$ as the trace $t$. The reduced model
$R(\EuScript{S}_i^{N})$ was previously constructed to capture
all these traces in $Traces_{Pass}(\EuScript{S}_i^{N})$, having
the same sequence of symbols. Indeed, given a STS
$\EuScript{S}_i^{N}$, all the STS paths of $\EuScript{S}_i^{N}$,
which have the same sequence of symbols labelled on the
transitions, are compacted into one STS path $b$ in
$R(\EuScript{S}_i^{N})$ whose transition guards are stored into a
matrix $M_{[b]}$.

If, we take back the trace example
$t=(17011(nsys=1,nsec=8,point=1,pid=1)\text{ }
17021(nsys=1,nsec=8,point=4,tpoint=9,pid=1))$, and the STS of
Figure \ref{fig:reduced-model}, $t$ is a pass trace w.r.t.
$\leq_{mct}$ because each assignment $\alpha_j(x)$ satisfies at
least one guard of the matrix line $j$. For instance, the
assignment $point=4$, which is given with the second valued
action of $t$, satisfies one of the guards of the second line of
the matrix $M_{[b]}$.

Given a trace $a_1(\alpha_1) \dots a_m(\alpha_m) \in CTraces({Sut})$
and a STS path $b$ of $R(\EuScript{S}_i^{N})$ having the same
sequence of symbols $a_1 \dots a_m$, the relation can be now
formulated as follows: for every valued action $a_j(\alpha_j)$,
each variable assignment $\alpha_j(x)$ must satisfies at least
one of the guards of the matrix line $j$ in $M_{[b]}[j,*]$.

The implementation relation $\leq_{mct}$ can then be written with:

\begin{proposition}
	${Sut} \leq_{mct} \EuScript{S}^{N} \text{ iff } \forall t=
	a_1(\alpha_1) \dots a_m(\alpha_m) \in CTraces({Sut}), \exists
	R(\EuScript{S}_i^{N}) \in R(\EuScript{S}^{N)} \text{ and }
	b=l0_{R(\EuScript{S}_i^{N})} \xRightarrow{(a_1(p_1),M_{[b]}[1,c_{[b]}]),\dots,(a_j(p_j),M_{[b]}[j,c_{[b]}])} l_m$ with $(1 \leq c_{[b]} \leq k)$ such that $\forall \alpha_j(x) (1 \leq j \leq m), \alpha_j(x) \models M_{[b]}[j,1] \vee \dots \vee  M_{[b]}[j,k]$ and $l_m \in Pass$.
\end{proposition}

% \begin{comment}
% This second implementation relation, denoted with $\leq_{mct}$, can be written with the reduced model
% $R(\EuScript{S}^{N})=\{R(\EuScript{S}_1^{N}),\dots,R(\EuScript{S}_n^{N})\}$. Indeed, given a STS $\EuScript{S}_i^{N}$, all the STS paths of$\EuScript{S}_i^{N}$, which have the same sequence of symbols labelled on the transitions, are compacted into one STS path $b$ in $R(\EuScript{S}_i^{N})$ whose transition guards are stored into a matrix $M_{[b]}$. Given a trace $a_1(\alpha_1)...a_m(\alpha_m) \in CTraces({Sut})$ and a STS path $b$ of $R(\EuScript{S}_i^{N})$ having the same sequence of symbols $a_1...a_m$, the relation can
% be now formulated as follows: for every valued action
% $a_j(\alpha_j)$, each variable assignment $\alpha_j(x)$ must
% satisfies at least one of the guards of the matrix line $j$ in
% $M_{[b]}[j,*]$.
%
%
%
%
% If, we take back the trace example
% $t=(17011(nsys=1,nsec=8,point=1,pid=1)\text{ }
% 17021(nsys=1,nsec=8,point=4,tpoint=9,pid=1)$, and the STS of
% Figure \ref{fig:reduced-model}, the assignment $point=4$, which
% is given with the second valued action of $t$, satisfies one of
% the guards of the second line of the matrix $M_{[b]}$. Hence, $t$ represents a correct behaviour w.r.t. the STS of
% Figure \ref{fig:reduced-model}.
%
% \begin{definition}
%     ${Sut} \leq_{mct} \EuScript{S}^{N} \text{ iff } \forall t=
%     a_1(\alpha_1)...a_1(\alpha_m),\\ \exists
%     R(\EuScript{S}_i^{N}) \text{ and } l0_{R(\EuScript{S}_i^{N})} \xRightarrow{(a_1(p_1),G_{1}),...,(a_m(p_m),G_{m})} l_m$ such that $\forall \alpha_j(x) (1 \leq j \leq m), \alpha_j(x) \models M_{[b]}[j,1] \vee ... \vee  M_{[b]}[j,k]$ and $l_m \in Pass$.
% \end{definition}
% \end{comment}

The disjunction of guards $M_{[b]}[j,1] \vee \dots \vee
M_{[b]}[j,k]$, found in the matrix $M_{[b]}$, could be simplified
by gathering all the equalities $x==val$ together with
disjunctions for every variable $x$ that belongs to the parameter
set $p_j$. Such equalities can be extracted with the
\textit{Proj} operator (see Definition \ref{def:sts}). We
obtain one guard of the form $\bigwedge_{x \in p_j}(x==val_1 \vee
\dots \vee x==val_k)$. The STS $D(\EuScript{S}_i^{N})$, derived
from $R(\EuScript{S}_i^{N})$, is constructed with this
simplification of guards:

%all the variable assignments found in the guards
%$M_{[b]}[j,1],..., M_{[b]}[j,k]$ can be collected with the
%projection operator $Proj_{x}(G)$.

%It is then possible to create
%vectors of guards from the matrices in $R(\EuScript{S}_i)^{G}$ in
%order to express that a trace $t$ can still comply with the
%behaviours found in $R(\EuScript{S}_i)^{G}$, even if it does not
%strictly satisfy all guards of a single branch (i.e. the same
%column in the matrices), but rather the disjunction of the guards
%belonging to the matrices of this branch.

\begin{definition}[Derived STS $D(\EuScript{S}_i^{N})$]
    Let $R(\EuScript{S}_i^{N})=<L_R,l0_R,V_R,V0_R,I_R,\Lambda_R,\rightarrow_R>$ be a STS of $R(\EuScript{S}^{N})$. We denote $D(\EuScript{S}_i^{N})$ the STS $ <L_D,l0_D,V_D,V0_D,I_D,\Lambda_D,\rightarrow_D>$ derived from $R(\EuScript{S}_i^{N})$ such that:
\begin{itemize}
    \item $L_D=L_{R}, l0_D=l0_{R}, I_D=I_{R}, \Lambda_D=\Lambda_{R}$,
    \item $V_D, V0_D$ and $\rightarrow_D$ are given by the following inference rule:

				$\frac{
					\begin{matrix}
					b=l0_{R}
					\xRightarrow{(a_1(p_1),M_{[b]}[1,c_{[b]}])\dots (a_m(p_m),M_{[b]}[m,c_{[b]}]}_{\rightarrow_{R}}
					l_{m},
					(1 \leq c_{[b]} \leq k) \text{ in } V0_{R}
					\end{matrix}
				}
				{
					\begin{matrix}
					l0_D
					\xRightarrow{(a_1(p_1),M_b[1])\dots (a_m(p_m),M_b[m])}_{\rightarrow_D}
					l_m\\
					V_D=V0_D \wedge M_b, M_b[j, 1]_{(1\leq j \leq m)} =\\
					 \displaystyle \bigwedge_{x \in p_j} ( Proj_x(M_{[b]}[j, 1]) \vee \dots \vee Proj_x(M_{[b]}[j, k]))


					%\bigvee_{1 \leq k \leq k} Proj_x ((M_{[b]}[j, k])

					\end{matrix}
				}$
  \end{itemize}

    $D(\EuScript{S}^{N})$ denotes the model $\{D(\EuScript{S}_1^{N}),\dots,D(\EuScript{S}_n^{N}) \}$.
\end{definition}


The second implementation relation $\leq_{mct}$ can now be
expressed by:

\begin{proposition}
    ${Sut} \leq_{mct} \EuScript{S}^{N} \text{ iff }\forall t=
    a_1(\alpha_1) \dots a_m(\alpha_m) \in CTraces({Sut}), \exists
    D(\EuScript{S}_i^{N}) \in D(\EuScript{S}^{N}) \text{ and }
    l0_{D(\EuScript{S}_i^{N})}
    \xRightarrow{(a_1(p_1),G_{1}),\dots,(a_m(p_m),G_{m})} l_m
    \text{ such that } \forall \alpha_j (1 \leq j \leq m),
    \alpha_j \models G_{j}$ and $l_m \in Pass$.
\end{proposition}


$\leq_{cmt}$ now means that a trace of $\mathit{Sut}$ must also
be a pass trace of the model
$D(\EuScript{S}^{N})=(D(\EuScript{S}_1^{N}),\dots,D(\EuScript{S}_n^{N}))$.
Furthermore, this notion of trace inclusion can be formulated
with the first implementation relation $\leq_{ct}$ as follows:

\begin{proposition}
	\label{rel:impl2}
${Sut} \leq_{mct} \EuScript{S}^{N} \text{ iff } CTraces({Sut})\subseteq Traces_{Pass}(D(\EuScript{S}^{N}))$

${Sut} \leq_{mct} \EuScript{S}^{N}\Leftrightarrow {Sut} \leq_{ct} D(\EuScript{S}^{N})$

\end{proposition}

The implementation relation $\leq_{mct}$ is now expressed with
the first relation $\leq_{ct}$, which implies that our passive
testing algorithms shall be the same for both relations but shall
take different reference models.
In the next section, we introduce an offline passive testing
algorithm that uses these two implementation relations.
