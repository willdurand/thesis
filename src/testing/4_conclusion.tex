\section{Conclusion}
\label{sec:testing:conclusion}

In this chapter, we presented a fast passive testing framework
built on-top of our model inference framework \emph{Autofunk v3},
which combines different techniques such as model inference,
expert systems, and machine learning.

In this work, we focus on complete traces exclusively because
production systems run continuously, and a few irrelevant
behaviors are likely to happen, \emph{e.g.}, collecting traces
can be turned either on or off at any time, but also human
operators in a factory can act on the products. Tracking such
irrelevant behaviors would be inefficient as it would mean more
processing time for results of no interest.

Given a large set of production events, our framework infers
exact models whose traces are included in the initial trace set
of a system under analysis. Such models are then reused as
specifications to perform offline passive testing using a second
set of traces recorded on a system under test. Using two
implementation relations based on complete trace inclusion,
\textit{Autofunk} is able to determine what has changed between
the two systems. This is particularly useful for our industrial
partner Michelin because potential regressions can be detected
while deploying changes in production. Initial results on this
offline method are encouraging, and Michelin engineers see a real
potential in this framework.

Given the preliminary results, We know that 2\% of a large trace
set (as mentioned in the previous section) still represents many
traces, which may be difficult to analyze. Nonetheless in our
manufacturing context, this is still valuable because, before
\emph{Autofunk}, engineers had to manually test everything by
hand. Now, \emph{Autofunk} performs most of the work
automatically, and engineers only have to manually check a small
subset of traces compared to the initial trace set, which saves a
lot of time.

In the next chapter, we give our thoughts on how to improve
\textit{Autofunk} as well as perspectives for future work.
