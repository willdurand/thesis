\section{Conclusion}
\label{sec:testing:conclusion}

In this chapter, we presented a fast passive testing framework
built on-top of our model inference framework \emph{Autofunk v3},
which combines different techniques such as model inference,
expert systems, and machine learning.

Given a large set of production events, our framework infers
exact models whose traces are included \cite{petrenko06} in the
initial trace set of a system under analysis. Such models are
then reused as specifications to perform: (i) offline passive
testing using a second set of traces recorded on a system under
test, and (ii) online passive testing by taking new traces of a
system under test on-the-fly. Using two implementation relations,
\textit{Autofunk} is able to determine what has changed between
the two systems. This is particularly useful for our industrial
partner Michelin because potential regressions can be detected
while deploying changes in production. Initial results on the
offline method are encouraging, and Michelin engineers see a real
potential in this framework.

We know that 2\% of a large trace set (as mentioned in the
previous section) still represents many traces, which may be
difficult to analyze. Nonetheless in a manufacturing context,
this is still valuable although we would like to reduce such
false negatives.

In the next chapter, we give our thoughts on how to improve
\textit{Autofunk} as well as perspectives for future work.
