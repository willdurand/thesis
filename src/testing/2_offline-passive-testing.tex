\subsection{Offline passive testing algorithm}
\label{sec:testing:offline}

Our offline passive testing algorithm, which aims to check
whether the two previous implementation relations hold, is given
in Algorithm \ref{algo:check}. It takes the complete traces of
$\mathit{Sut}$, as well as the model sets $R(\EuScript{S}^{N})$ and
$D(\EuScript{S}^{N})$, with regard to Proposition
\ref{rel:impl12} and Proposition \ref{rel:impl2}. It returns the
verdict "Pass$\leq_{ct}$" if the relation $\leq_{ct}$ is
satisfied, and the verdict "Pass$\leq_{mct}$" if $\leq_{mct}$ is
satisfied.

Algorithm \ref{algo:check} relies upon the function ${check(Trace
~trace, STS ~\EuScript{S})}$ to check whether the trace $trace =
(a_1(p), \alpha_1) \dots (a_m(p), \alpha_m)$ is a trace of
$\EuScript{S}$. If a STS path $b$ is composed of the same
sequence of symbols than $trace$ (line
\ref{algo:check:line:exists}), the function tries to find a
matrix column (\emph{i.e.} a vector) $M = M_{[b]}[*,c_{[b]}]$ ($1
\leq c_{[b]} \leq k$) such that every variable assignment
$\alpha_j$ satisfies the guard $M[j]$. If such a column of guards
exists, the function returns $True$, and $False$ otherwise (cf.
Proposition \ref{prop:check}).

First, this algorithm covers every trace $trace$ of
$CTraces({Sut})$, and tries to find a STS $R(\EuScript{S}_i^{N})$
such that $trace$ is also a trace of $R(\EuScript{S}_i^{N})$ with
$check(trace, R(\EuScript{S}_i^{N}))$ (line
\ref{algo:check:line:check1}).  If no model
$R(\EuScript{S}_i^{N})$ is found, $trace$ is added to the set
$T_1$ (line \ref{algo:check:line:t1}), which gathers the possibly
fail traces with respect to $\leq_{ct}$.
Thereafter, this algorithm performs the same step but using the
STS $D(\EuScript{S}^{N})$ (line \ref{algo:check:line:check2}).
One more time, if no model $D(\EuScript{S}_i^{N})$ is found, the
trace $trace$ is added to the set $T_2$ (line
\ref{algo:check:line:t2}), which gathers the possibly fail
traces with respect to the relation $\leq_{mct}$.
Finally, if $T_1$ is empty, the verdict "Pass$\leq_{ct}$" is
returned, which means that the first implementation relation
holds. Otherwise, $T_1$ is provided. If $T_2$ is empty, the
verdict "Pass$\leq_{mct}$" is returned. Otherwise $T_2$ is
returned.

\begin{algorithm}[h]
    \SetKwInOut{Input}{Input}
    \SetKwInOut{Output}{Output}
    \SetKwFunction{check}{check}

    \Input{
        $R(\EuScript{S}^{N}),
        D(\EuScript{S}^{N}),
        CTraces({Sut})$
    }
    \Output{Verdicts and/or possibly fail trace sets $T_1, T_2$ }

    BEGIN\;

    $T_1 = \emptyset$\;
    $T_2 = \emptyset$\;

    \ForEach{$trace \in CTraces({Sut})$}{

        \For{$i = 1, \dots, n$}{\label{algo:check:line:proof1-start}
            \If{\check($trace$, $R(\EuScript{S}_i^{N})$)}{\label{algo:check:line:check1}
                break\;
            }
        }\label{algo:check:line:proof1-end}

        \If{$i == n$} {
            $T_1=T_1 \cup \{trace\}$\;\label{algo:check:line:t1}

            \For{$i = 1, \dots, n$}{\label{algo:check:line:proof2-start}

                \If{\check($trace$, $D(\EuScript{S}_i^{N})$)}{\label{algo:check:line:check2}
                    break\;
                }
            }\label{algo:check:line:proof2-end}

            \If{$i==n$} {
                $T_2=T_2 \cup \{trace\}$\;\label{algo:check:line:t2}
            }
        }
    }%endfor1

    \BlankLine

    \If{$T_1==\emptyset$}{\label{algo:check:line:empty-t1}
        return "Pass$\leq_{ct}$"\;\label{algo:check:line:pass_ct}
    }
    \Else{
        \If{$T_2==\emptyset$}{\label{algo:check:line:empty-t2}
            return "Pass$\leq_{mct}$" and $T_1$\;\label{algo:check:line:pass_mct}
        }
        \Else{
            return $T_1$ and $T_2$\;
        }
    }

    END\;

    \BlankLine
    \BlankLine

    \SetKwProg{Fn}{Function}{ is}{end}
    \Fn{check(Trace $trace$, STS $\EuScript{S}$) : boolean}{
        \If{$\exists ~b=l0_{\EuScript{S}}
            \xRightarrow{(a_1(p_1),G_{1},A_{1}) \dots
            (a_n(p_n),G_{n},A_{n})} l_n \mid trace =
            (a_1(p), \alpha_1)\dots (a_n(p), \alpha_n)$ and $l_n \in
            Pass$}{\label{algo:check:line:exists}

                $M_{[b]} = Mat(b)$ is the matrix $n \times k$ of $b$\;
                $c_{[b]} = 1$\;
                \While{$c_{[b]} \leq k$}{
                    $M = M_{[b]}[*,c_{[b]}]$\;

                    \For{$j = 1, \dots, n$}{
                        \If{$\alpha_j \not\models M[j]$}{
                            break\;
                        }
                    }

                    \If{$j == n$}{
                        return $True$\;
                    }
                    $c_{[b]}++$\;
                }
        }
        return $False$\;
    }

    \caption{Offline passive testing algorithm}
    \label{algo:check}
\end{algorithm}

When one of the implementation relations does not hold, this
algorithm offers the advantage of providing the possibly fail
traces of $CTraces({Sut})$. Such traces can be later analyzed to
check if $\mathit{Sut}$ is correct or not. That is very helpful
for Michelin engineers because it allows them to only focus on
what are potentially faulty behaviors, reducing debugging time,
and making engineers more efficient.

\begin{proposition}
    Let $t \in CTraces(Sut)$ be a trace, and $\EuScript{S}^{N}$ a
    STS set such that $\mathit{Sut} \leq_{ct} \EuScript{S}^{N}$.
    $\leq_{ct}$ means that there exists a model
    $\EuScript{S}_i^{N}$ such that $t \in
    Traces_{Pass}(\EuScript{S}_i^{N})$, \emph{i.e.} $t$
    \emph{conforms to} $\EuScript{S}_i^{N}$.

    The $check(Trace ~trace, STS ~\EuScript{S})$ function in
    Algorithm \ref{algo:check} implements this relation, and we
    have: $t \in Traces_{Pass}(\EuScript{S}_i^{N}) \implies$ the
    function $check(t, \EuScript{S}_i^{N})$ returns $True$.

    \begin{proof}
        \textbf{Sketch of proof:} Let $t = (a_1(p), \alpha_1)
        \dots (a_n(p), \alpha_n)$ be a trace of $\mathit{Sut}$.
        Let $b$ be a STS path such that
        $\exists ~b=l0_{\EuScript{S}_i^{N}} \xRightarrow{(a_1(p_1),G_{1},A_{1}) \dots
        (a_n(p_n),G_{n},A_{n})} l_n, ~l_n \in Pass \mid ~t \in Traces_{Pass}(b)$.

        Given the trace $t = (a_1(p), \alpha_1) \dots (a_n(p),
        \alpha_n)$, the function $check$:

        \begin{itemize}
            \item finds a path $b = l0_{\EuScript{S}_i^{N}}
                \xRightarrow{(a_1(p_1),G_{1},A_{1}) \dots
                (a_n(p_n),G_{n},A_{n})} l_n, ~l_n \in Pass$ (line
                \ref{algo:check:line:exists}), with $M_{[b]}$ be
                the matrix $n \times k$ of $b$ such that $G_j =
                M_{[b]}[j, c_{[b]}] ~(1 \leq c_{[b]} \leq k$ and
                $1 \leq j \leq n)$;

            \item ensures that $\exists ~1 \leq c_{[b]} \leq k
                \mid \alpha_j \models M_{[b]}[j, c_{[b]}] ~(1
                \leq j \leq n)$ (lines 28-31). %TODO: fixme
        \end{itemize}

        If both a path $b$ and a column $c_{[b]}$ of $M_{[b]}$
        are found, $True$ is returned (line 33).
    \end{proof}

    \label{prop:check}
\end{proposition}

\paragraph{Soundness.} The soundness of Algorithm
\ref{algo:check} can be split into three points (sketch of proof):

\begin{enumerate}
    \item $\mathit{Sut} \leq_{ct} \EuScript{S}^N$ $\implies$
        Algorithm \ref{algo:check} returns "Pass$\leq_{ct}$":

        \begin{proof}
            $\mathit{Sut} \leq_{ct} \EuScript{S}^{N}
            \Leftrightarrow CTraces(Sut) \subseteq Traces_{Pass}(R(\EuScript{S}^N))$
            (Proposition \ref{rel:impl12}) can be written as
            follows: $\forall ~t \in CTraces(Sut), \exists ~1
            \leq i \leq n \mid ~t \in Traces_{Pass}(R(\EuScript{S}_i^{N}))$.

            Given that, and according to Proposition
            \ref{prop:check}, function $check$ returns $True$ for
            every trace $t \in CTraces(Sut)$ (lines
            \ref{algo:check:line:proof1-start}-\ref{algo:check:line:proof1-end}).

            Therefore the set $T_1$ is empty (line
            \ref{algo:check:line:empty-t1}), and Algorithm
            \ref{algo:check} returns "Pass$\leq_{ct}$" (line
            \ref{algo:check:line:pass_ct}).
        \end{proof}

    \item $\mathit{Sut} \leq_{mct} \EuScript{S}^N$ $\implies$
        Algorithm \ref{algo:check} returns "Pass$\leq_{mct}$":

        \begin{proof}
            $\mathit{Sut} \leq_{mct} \EuScript{S}^{N}
            \Leftrightarrow CTraces(Sut) \subseteq Traces_{Pass}(D(\EuScript{S}^N))$
            (Proposition \ref{rel:impl2}) can be written as
            follows: $\forall ~t \in CTraces(Sut), \exists ~1
            \leq i \leq n \mid ~t \in Traces_{Pass}(D(\EuScript{S}_i^{N}))$.

            Given that, and according to Proposition
            \ref{prop:check}, function $check$ returns $True$ for
            every trace $t \in CTraces(Sut)$ (lines
            \ref{algo:check:line:proof2-start}-\ref{algo:check:line:proof2-end}).

            Therefore the set $T_2$ is empty
            (line \ref{algo:check:line:empty-t2}), and Algorithm
            \ref{algo:check} returns "Pass$\leq_{mct}$" (line
            \ref{algo:check:line:pass_mct}).
        \end{proof}

    \item $\mathit{Sut} \leq_{ct} \EuScript{S}^N$ $\implies$
        Algorithm \ref{algo:check} returns "Pass$\leq_{ct}$",
        "Pass$\leq_{mct}$":

        \begin{proof}
            $\mathit{Sut} \leq_{ct} \EuScript{S}^{N} \implies
            \mathit{Sut} \leq_{mct} \EuScript{S}^{N}$
            (Proposition \ref{rel:impl-ct-implies-mct}), therefore
            Algorithm \ref{algo:check} returns "Pass$\leq{ct}$",
            "Pass$\leq_{mct}$".
        \end{proof}
\end{enumerate}

\paragraph{Complexity.}
The complexity of the function $check(t, \EuScript{S})$ is
$\mathcal{O}(m \times k)$ with $m$ the number of valued events in
the trace $t$ (\emph{i.e.} its length), and $k$ the number of
columns in $M_{[b]}$, which is likely large as reduced models
still express all complete behaviors found in the traces of a
system under analysis. Finding a branch in a model is negligible
thanks to the hash mechanism, hence we only take the matrix
traversal into account. Because $m << k$, the complexity of the
function $check(t, \EuScript{S})$ can be simplified to
$\mathcal{O}(k)$.

The complexity of Algorithm \ref{algo:check} is $\mathcal{O}(t
\times n \times k)$ with $t$ the number of complete traces of
$\mathit{Sut}$, $n$ the number of models, and $k$ the complexity
of the $check$ function, which corresponds to the number of
columns of the matrix associated with the branch (as described
above). Compared to the number of traces and columns, the number
of models $n$ is negligible (\emph{i.e.} $n << t$, $n << k$, and
$t \approx k$), which means that the overall complexity is
$\mathcal{O}(t^2)$.

\clearpage
