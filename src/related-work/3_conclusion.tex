\section{Conclusion}
\label{sec:related:conclusion}

As stated previously, the use of active inference techniques is
not possible in our industrial context. Production systems are
both distributed and heterogeneous event-driven systems, compound
of software and several physical devices (\emph{e.g.}, points,
switches, stores, and production machines). Applying active
inference on them without disturbing them is therefore not
feasible in practice. We are not supposed to cause damages on
these systems while studying them, that is why we investigated
the existing passive inference techniques.

Given the constraints formulated in
\crossref{sec:intro}{sec:intro:problems}, we are not able to
leverage white-box or documentation-based works. For the record,
we cannot safely reuse existing documentation, and due to the
heterogeneous set of software we target, we cannot rely on
white-box approaches either. At first glance, and because
production systems are event-driven, techniques using event
sequence abstraction seemed the most relevant to us.

Yet, our main concern regarding methods such as kTail and
kBehavior was the over-approximation tied to the inferred models.
Indeed, while it may not always be an issue, depending on the use
cases, our context requires exact models for testing. That is why
we present two new approaches combining passive model inference,
machine learning, and expert systems to infer models from traces
for web applications (Chapter \ref{sec:modelinf:webapps}) and
industrial systems (Chapter \ref{sec:modelinf:prodsystems}) in
the sequel.  The second technique is an adaptation of the first
one.

We believe that knowledge of human domain experts is valuable,
hence the use of expert systems to integrate their knowledge with
our inference techniques. Expert systems are computer systems
that emulate the decision-making abilities of humans. We
"transliterate" knowledge of domain experts into inference rules,
which our inference techniques leverage thanks to expert systems.
Our work is still similar to those using event sequence
abstraction (page \pageref{sec:passive-fsa}), except that state
merging is replaced with a context-specific state reduction based
on an event sequence abstraction. This state reduction can be
seen as the kTail algorithm introduced previously where $k$ would
be dynamic and as high as possible.
Furthermore, our two simple yet scalable techniques infer exact
models for testing purpose. We focus on both speed and
scalability to be able to construct models of Michelin's
production systems in an efficient manner, which is often a
must-have for adoption in the Industry.

Among all potential use cases, we need such models to perform
passive testing on these production systems (Chapter
\ref{sec:testing}). We chose to perform passive testing for the
same reasons mentioned before, \emph{i.e.} because passive
testing does not disturb the system, and because it can be
applied to large and heterogeneous systems.
