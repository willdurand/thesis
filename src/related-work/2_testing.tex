\section{Software testing}
\label{sec:related:testing}

Software testing is the process of executing a program or system
with the intent of finding errors \cite{Myers:1979:AST:539883}.
However, testing shows the presence, not the absence of bugs as
Edsger Wybe Dijkstra used to say.

Testing is achieved by analyzing a software to detect the
differences between existing and required conditions (that is,
bugs) and to evaluate the features of this software. As Meyers
said, testing is used to find faults, but it is also useful to
provide confidence of reliability, correctness, and absence of
particular faults on software we develop. This does not mean that
the software is completely free of defects. Rather, it must be
good enough for its intended use.
Testing is a verification and validation process
\cite{wallace1989software}. I use to (informally) explain both
terms with the following questions, although it is worth
mentioning that I am certainly not the author of any of them:

\begin{itemize}
\item \textbf{Validation:} are we building the right software?
\item \textbf{Verification:} are we building the software right?
\end{itemize}

In other words, formal verification is the act of proving or
disproving the correctness of intended algorithms underlying a
system with respect to a certain formal specification or
property, using formal methods of mathematics. Model checking,
runtime verification, theorem proving, static analysis and
simulation are all verification methods.

Validation is testing as the act of revealing bugs. That is what
most people think testing is, and also the meaning we give to the
word "testing" in the sequel of this thesis.
Nowadays, software testing, if not always applied, is well-known
in the Industry. It is considered a good practice and many
techniques and tools have been developed over the last 10 years.
Most of them are different in nature and have different purposes.
There are a lot of new terms that all end with "Testing" such as:
Unit Testing, Integration Testing, Functional Testing, System
Testing, Stress Testing, Performance Testing, Usability Testing,
Acceptance Testing, Regression Testing, Beta Testing, and so on.

\begin{itemize}
\item \textbf{Unit Testing:}

\item \textbf{Integration Testing:}

\item \textbf{Acceptance Testing:}

\item \textbf{System Testing:}
\end{itemize}

The most positive fact about this is that everyone in the
Industry understands the need for testing, and tries to improve
his daily routine. However, testing is still mostly performed by
hand.

\subsection{}
