Ce manuscrit de thèse aborde le problème du test basé modèle de
systèmes de production existants, tels ceux de notre partenaire
industriel Michelin, l'un des trois plus grands fabricants de
pneumatiques au monde. Un système de production est composé d'un
ensemble de machines de production contrôlées par un ou plusieurs
logiciels au sein d'un atelier d'une usine. Malgré les nombreux
travaux dans le domaine du test basé modèle, l'écriture de
modèles permettant de décrire un système sous test ou sa
spécification reste un problème commun, en partie à cause de la
complexité d'une telle tâche. De plus, un modèle est utile
lorsqu'il est à jour par rapport à ce qu'il décrit, ce qui
implique de le maintenir à jour dans le temps. Pour autant,
conserver une documentation à jour reste compliqué puisqu'il faut
souvent le faire manuellement. Dans notre contexte, il est
important de souligner le fait qu'un système de production
fonctionne en continu et ne doit pas être arrêté ni perturbé, ce
qui limite l'usage des techniques de test classiques.

Pour pallier le problème de l'écriture de modèles, nous proposons
une approche pour construire automatiquement des modèles depuis
des séquences d'événements observés (des traces) dans un
environnement de production. Pour se faire, nous utilisons les
informations fournies par les données échangées entre les
éléments qui composent un système de production. Nous adoptons
une approche boîte noire et combinons les notions de système
expert, inférence de modèles et \emph{machine learning}, afin de
créer des modèles comportementaux. Ces modèles inférés décrivent
des comportements enregistrés sur un système sous analyse. Ils
sont partiels, mais également très grands (en terme de taille),
ce qui les rend difficilement utilisable par la suite. Nous
proposons une technique de réduction spécifique à notre contexte
qui conserve l'équivalence de traces entre les modèles de base et
les modèles fortement réduits. Grâce à cela, ces modèles inférés
deviennent intéressant pour la génération de documentation, la
fouille de données, mais également le test.

Nous proposons une méthode passive de test basé modèle pour
répondre au problème du test de systèmes de production sans
interférer sur leur bon fonctionnement. Cette technique permet
d'identifier des différences entre deux systèmes de production et
réutilise l'inférence de modèles décrite ci-avant. Nous
introduisons deux relations d'implantation : une relation basée
sur l'inclusion de traces, et une seconde relation plus faible
proposée pour remédier au fait que les modèles inférés soient
partiels.

De manière générale, ce manuscrit de thèse présente
\emph{Autofunk}, un \emph{framework} modulaire pour l'inférence
de modèles et le test de systèmes de production qui aggrège les
notions mentionnées précédemment. Son implémentation en Java a
été appliquée sur différentes applications et systèmes de
production chez Michelin dont les résultats sont donnés dans ce
manuscrit. Le prototype développé lors de la thèse a pour
vocation de devenir un outil standard chez Michelin.
