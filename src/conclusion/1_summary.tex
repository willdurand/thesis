\section{Summary of achievements}

This thesis has proposed a novel approach to infer models of
software systems in order to perform conformance testing of
production systems, with a technique that leverages such models.
The original aims and objectives of the thesis were as follows:

\begin{itemize}
    \item To infer partial yet exact models of production systems
        in a fast and efficient manner, based on the data
        exchanged in a (production) environment;

    \item To design a conformance testing technique based on the
        inferred models, targeting production systems. The main
        idea was to detect regressions across similar production
        systems (\emph{e.g.}, a software update or a hardware
        upgrade).
\end{itemize}

Chapters \ref{sec:modelinf:webapps} and
\ref{sec:modelinf:prodsystems} addressed the first of the
objectives by proposing two approaches combining model inference,
machine learning, and expert systems to infer exact models for
web applications and production systems, wrapped into the
\textit{Autofunk} framework.  The expert system is composed of
rules, capturing the knowledge of human experts, and used either
to filter the trace set to remove the undesired ones, to infer
Symbolic Transition Systems (STSs), or to build
more abstract STSs. In Chapter \ref{sec:modelinf:prodsystems},
the state merging is replaced with a context-specific state
reduction based on an event sequence-based abstraction. This state
reduction can be seen as the kTail algorithm \cite{5009015} where
$k$ is as high as possible for every initial branch (or path) of
the original Symbolic Transition System (STS). This state
reduction ensures that the resulting models do not
over-approximate the system under analysis, but it is also very
context-specific, and cannot be generalized. We also showed that
our approach is scalable: it can take thousands and thousands of
traces and can still build models quickly thanks to our specific
state merging process.

The second objective was achieved by enhancing \textit{Autofunk}
with a passive testing technique, presented in Chapter
\ref{sec:testing}. Given a large set of production events,
\textit{Autofunk} reuses the inferred models as specifications to
perform offline passive testing, using a second set of traces
recorded on a system under test, and two implementation relations
to determine what has changed between the two systems. This is
particularly useful for our industrial partner Michelin because
potential regressions can be detected while deploying changes in
production.

The next section introduces some perspectives for future work on
model inference. Section \ref{sec:conclusion:testing} is
dedicated to future work on the testing part of our work. Section
\ref{sec:conclusion:final-thoughts} closes this thesis.
